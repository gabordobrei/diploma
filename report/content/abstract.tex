%----------------------------------------------------------------------------
% Kivonat
%----------------------------------------------------------------------------
\chapter*{Kivonat}\addcontentsline{toc}{chapter}{Kivonat}

Az Internet rohamos fejlődésével az utóbbi időben mind nagyobb hangsúlyt kap a hagyományostól eltérő hálózatmenedzsment funkciókat megvalósító algoritmusok kutatása. Ezek egyik legaktívabban művelt ága az útvonalválasztás (routing) kérdéseivel foglalkozik, és lényegében az erőforrások optimalizálást célzó klasszikus (főként legrövidebb utak megtalálására koncentráló) algoritmusok alternatíváinak kutatását tűzi ki célul.\\

Az algoritmusok általános szabályok (policy) mentén alakítják ki a lehetséges útvonalakat. A modern hálózatkutatás még nem adott megoldást arra a problémára, hogy hogyan deríthetjük fel egy hálózat szabályrendszerét a kialakított utak megfigyelésével. Az útvonalválasztási szabályok matematikai modellezése megfelelő alapot ad \cite{Compact_Policy_Routing} egy olyan rendszer kifejlesztéséhez, amivel erre a problémára egy hiánypótló megoldást adhatunk.\\

Ebben a Diplomatervben leírok egy olyan szimulációs keretrendszert, amely képes a nagyméretű valós hálózatok útvonalválasztási szabályrendszerét vizsgálni és alkalmasan választott új szabályokkal akár javítani is. A szimulátor működését be is mutatom egy konkrét hálózat elemzésével: a repülőtársaságok útvonalválasztási szabályrendszerét azonosítom.

\vfill

\iffalse
%----------------------------------------------------------------------------
% Abstract
%----------------------------------------------------------------------------
\chapter*{Abstract}\addcontentsline{toc}{chapter}{Abstract}

This document is a \LaTeX-based skeleton for BSc/MSc~theses of students at the Electrical Engineering and Informatics Faculty, Budapest University of Technology and Economics. The usage of this skeleton is optional. It has been tested with the \emph{TeXLive} \TeX~implementation, and it requires the PDF-\LaTeX~compiler.
\vfill
\fi