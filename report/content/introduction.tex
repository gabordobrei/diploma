%----------------------------------------------------------------------------
\chapter*{Bevezető}\addcontentsline{toc}{chapter}{Bevezető}
%----------------------------------------------------------------------------
Az Internet rohamos fejlődésével egyre nagyobb igény mutatkozik a komplex számítógépes hálózatok megismerésére. Ha egy hálózatban ismerjük a csomópontokat, az összeköttetéseket és az útvonalválasztást meghatározó szabályokat, akkor bármilyen módosítást egyszerűen el tudunk végezni, az esetleges hibákat azonnal ki tudjuk javítani. Változtathatunk az útvonalválasztási szabályokon, vagy implementálhatjuk azokat egy új hálózatmenedzsment megoldásban (lásd SDN\footnote{Software-Defined Networking: Szoftver-Definiált Hálózat} bevezetése\cite{sdn-deploy}). Olyan hálózat vizsgálata során, aminek nem ismerjük az útvonalválasztási szabályait, nem tudjuk elvégezni ezeket az alapvető feladatokat, sőt olyan problémákkal szembesülünk, amiket csak sok munka árán tudunk megválaszolni -- mérésekkel, teszteléssel -- és csak abban az esetben, ha a hálózat mérete még nem jelent problémát.\\

Az útvonalakat meghatározó algoritmusok általános szabályok (policy) mentén alakítják ki a lehetséges (kommunikációs) útvonalakat, és bár a kapcsolódó irányadó paraméterek igen változatosak lehetnek, a problémakör egy meghatározó paramétere szerinti legrövidebb útvonalat szoktuk a legjobbnak tekinteni. Mégis, ha megvizsgáljuk az Internet magas szintű topológiáját\footnote{Nevezik még az Internet tartomány-szintű topológiájának, gráfjának is.} és a benne kialakult utakat, akkor azt vesszük észre, hogy ezek az utak nem mindig a legrövidebbek. Mivel feltehetjük, hogy az Autonóm Rendszerek\footnote{Autonóm Rendszer -- Autonomous System (AS): önálló útválasztási tartomány, amelyen belül egyetlen, jól meghatározott útvonalválasztási szabály érvényesül.} üzemeltetői -- gazdasági szereplőként -- az anyagi haszon maximalizálása céljából racionális döntések révén építették ki pontosan ezeket az utakat, jogos kérdés, hogy pontosan milyen stratégia alapján tették ezt. Mi vezethet egy látszólag ésszerűtlen döntéshez? Ezt a látszólagos ellentmondást oldhatjuk fel, ha megismerjük azt a policy-t, ami az útvonalválasztást meghatározta.\\

Ha körbenézünk a világban, az informatikától távol eső területeken is rengeteg példát találunk olyan hálózatokra, amiknek nem értjük még a működését, nem tudjuk pontosan leírni a belső folyamatait. Számos példát találunk a biológiából, a szociológiából vagy a pénzügyi világból. Minden ilyen probléma vizsgálatát elvégezhetjük gráfokkal: a vírusok által terjesztett betegség terjedése, egy ruhaviselet divattá válása és a számítógépes hálózatok kommunikációs útjainak kialakulása is modellezhető a gráfelmélet eszközrendszerével. Mindhárom esetben jól meghatározható, elkülönülő egyedek vannak -- a csomópontok --, amik között kapcsolatokat értelmezünk -- az éleket. Egy modellen belül a csomópontok általában nem különböznek egymástól, a köztük levő kapcsolatok különböző tulajdonságokkal rendelkeznek, amit az útvonalak kialakulása közben figyelembe is kell venni. Pl. a vírusterjedéskor nem tételezünk fel különbséget két ember között, bárki meg tud betegedni, viszont az is magától értetődő, hogy cseppfertőzéssel terjedő vírus jóval ,,lazább'' kapcsolaton keresztül is továbbterjed, mint egy vér útján terjedő betegség. Az ilyen fajta megkülönböztetést a gráfoknál az élek súlyozásával érhetjük el. Ha egy ember fogékonyabb egy betegségre, akkor a hozzá tartozó éleken nagyobb valószínűséggel terjed tovább a betegség. Ezzel teljesen analóg módon, két szomszédos router között a nagyobb sávszélességű úton továbbítjuk a csomagokat. Érdekes azonban, hogy egy divat elterjedését már nem tudjuk ilyen egyszerűen leírni, hiszen nem az a meghatározó tényező egy termék elterjedésében, hogy mennyire befolyásos és ismert emberek reklámozzák, hanem az, hogy a társadalom felkészült-e már a befogadásra\cite{DuncanWatts, DobreiMScOnlab1}.\\

A Diplomatervben olyan egyszerű szabályokat -- policy primitíveket -- határozok meg, amelyek összekapcsolásával, egymás utáni alkalmazásával jól lehet közelíteni a valós hálózatokat. A számítógépes hálózatok területén sok a megválaszolatlan kérdés, pl. a BGP\footnote{Border Gateway Protocol -- Az Internet tartományszintű hálózatában történő routing protokollja.} hálózat útvonalválasztásáról. Ezek jó része onnan ered, hogy a hálózat csomópontjai, az AS-ek nem fedik fel szabályrendszerüket. Ezen kérdésekre tudnánk válaszolni úgy, ha felderítjük ezen szabályokat. Ennek egy módja, hogy nem az eredeti szabályokat adjuk meg, hanem megmutatjuk, hogy policy primitívek egy lehetőleg minimális halmazának felhasználásával majdnem ugyanolyan utakat kapunk, mint a megfigyelt, valós AS-beli utak. Ez a megoldás egy másik szempontból is hasznos lenne, hiszen így még az AS akár bonyolult policy-rendszerének egyszerűsítési lehetőségeit is feltárhatjuk.\\

\Aref{modell}. fejezetben a számítógépes hálózatok policy-felderítésével foglalkozó szakirodalom áttekintése után egy általános hálózati- és routing modellt írok le, amely képes kezelni más tudományterületekről származó hasonló problémákat is. Az útvonalválasztás szabályrendszerét egy jól definiált matematikai struktúrával kell meghatározni, hogy definiálni lehessen policy-k közti műveleteket, amelyekkel policy-ket kapcsolunk össze és tudjuk vizsgálni a kölcsönhatásukat.\\

\Aref{examples}. fejezetben változatos problémákat és a hozzájuk tartozó routing policy-ket definiálok. Leírom a hozzájuk tartozó matematikai struktúrákat, különös tekintettel az olyan szabályrendszerekre, amelyeket a későbbi szimulációs vizsgálatoknál fel tudok használni.\\

\Aref{framework}. fejezetben specifikálok egy olyan szimulációs keretrendszert, amelynek segítségével a valós életből származó, tipikusan nagyméretű hálózatok útvonalválasztási szabályrendszereit képes jellemezni. Ehhez kidolgozok egy alkalmas pontozási rendszert, amely segítségével a szimulációs eredményeket értékelni tudom. Ennek részeként beazonosítom azokat a legfontosabb globális- és útvonalfüggő metrikákat, amelyekkel a lehető legpontosabban tudom mérni egy szimuláció pontosságát, valamint azt, hogy milyen mértékben tudnánk javítani a megfigyelt hálózat útvonalválasztásán.\\

\Aref{test}. fejezetben a szimulációs keretrendszerrel elemzem, hogy egy valós hálózatban, a repülőtársaságok útvonalaiban milyen utakat határoznak meg \aref{modell}. fejezetből alkalmasan választott policy-k. Javaslatot teszek, hogy repülőtársaságok útvonalválasztását melyik policy-k keverékével lehet legpontosabban modellezni, illetve javítani.\\

A Diplomatervben általában a középiskolai matematikát meg nem haladó eszközrendszerrel dolgozom, az ennél komolyabb fogalmak, definíciók, tételek pedig megtalálhatók a \hyperlink{appendix}{Függelékben}. Mindazonáltal a számításelmélet, az algoritmuselmélet és egyéb magasabb szintű témakörök ismerete elengedhetetlen ennek a témának a tárgyalása során.
