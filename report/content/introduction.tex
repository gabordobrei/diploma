%----------------------------------------------------------------------------
\chapter*{Bevezető}\addcontentsline{toc}{chapter}{Bevezető}
%----------------------------------------------------------------------------

Az Internet rohamos fejlődésével egyre nagyobb igény mutatkozik a komplex számítógépes hálózatok megismerésére. Általános esetben a (kis) hálózatok struktúrája és működési mechanizmusai jól meghatározottak, hiszen a saját tervezésünk eredményeként jöttek létre és a vezérlés is a mi kezünkben van. Pontosan tudjuk, hogy egy hálózati csomópont melyik másik csomóponttal van kapcsolatban, ismerjük az összeköttetéseket és az útvonalválasztást meghatározó szabályokat is. Ha bármilyen módosítást szeretnénk eszközölni, vagy egy hibát szeretnénk kijavítani, azonnal (nagyon gyorsan) tudjuk, hogy hol kell beavatkozni.

Azonban egy olyan hálózat vizsgálata során, amelyet nem mi tervezetünk, nagyon gyorsan szembesülünk olyan kérdésekkel, amiket csak nehezen és sok munka árán tudunk megválaszolni -- mérésekkel, teszteléssel -- és csak abban az esetben, ha a hálózat mérete még nem jelent problémát.\\

Az útvonalakat meghatározó algoritmusok általános szabályok (policy) mentén alakítják ki a lehetséges (kommunikációs) útvonalakat és bár a kapcsolódó irányadó paraméterek igen változatosak lehetnek, a problémakör egy meghatározó paramétere szerinti legrövidebb útvonalat szoktuk a legjobbnak tekinteni. Mégis, ha megvizsgáljuk az Internet magas szintű topológiáját\footnote{Nevezik még az Internet tartomány-szintű-, vagy AS-szintű topológiájának, gráfjának is.} és a benne kialakult utakat, akkor azt vesszük észre, hogy ezek az utak nem az optimális megoldások, legalábbis nem a legrövidebbek. Mivel feltehetjük, hogy az anyagi haszon maximalizálása céljából az Autonóm Rendszerek\footnote{Autonóm Rendszer -- Autonomous System (AS): önálló útválasztási tartomány, amelyen belül egyetlen, jól meghatározott útvonalválasztási szabály érvényesül.} üzemeltetői racionális döntések révén építették ki pontosan ezeket az utakat, jogos kérdés, hogy pontosan milyen stratégia alapján tették ezt. Mi vezethet egy látszólag ésszerűtlen döntéshez? Általánosságban igaz az egyéb hálózatokra is, hogy az adott szituáció optimális útjai nem tűnnek racionálisnak.\\

Ha körbenézünk a világban, az informatikától távol eső területeken is rengeteg példát találunk olyan hálózatokra, amiknek nem értjük még a működését, nem tudjuk pontosan leírni a belső folyamatait. Számos példát találunk a biológiából, a szociológiából vagy a pénzügyi világból, ugyanakkor minden ilyen probléma vizsgálatát vissza lehet vezetni egy olyan modell vizsgálatára, ami általánosan képes kezelni magát a hálózat fogalmát és az azon történő útvonalak kialakulását / kialakítását. A vírusok által terjesztett betegség terjedése, egy ruhaviselet divattá válása, és a számítógépes hálózatok kommunikációs útjainak kialakulása modellezhető a gráfelmélet eszközrendszerével. Mindhárom esetben jól meghatározható, elkülönülő csomópontok vannak, akik között terjed egy csomag, ami lehet információ vagy pl. a betegség maga. Egy modellen belül a csomópontok általában nem különböznek egymástól, és a köztük levő kapcsolatok különböző tulajdonságokkal rendelkeznek, amit az útvonalak kialakulása közben figyelembe is kell venni. Nem tételezünk fel különbséget két ember között, bárki meg tud betegedni. Magától értetődőnek látszik ugyanakkor, hogy levegőben terjedő vírusos fertőzés jóval lazább kapcsolaton keresztül is továbbterjed, míg egy vér útján terjedő betegség sokkal szorosabb kapcsolaton tud csak továbbterjedni: a kapcsolatokat súlyozni kell. Ha egy ember fogékonyabb egy betegségre, akkor a hozzá tartozó éleken nagyobb valószínűséggel terjed tovább majd a betegség. Ugyanilyen meggondolásból, két szomszédos router között a nagyobb sávszélességű úton továbbítjuk a csomagokat. Érdekes azonban, hogy egy divat elterjedését már nem tudjuk ilyen egyszerűen leírni, hiszen nem is igazán az a meghatározó, hogy mennyire befolyásos és ismert emberek reklámozzák, hanem az, hogy a társadalom felkészült-e már a befogadásra \cite{DuncanWatts, DobreiMScOnlab1}.\\

A Diplomatervben a végső cél olyan policy primitíveket meghatározni, melyek összekapcsolásával jól lehet közelíteni a valós hálózatokat. Jelenleg sok megválaszolatlan kérdés van a BGP hálózat útvonalválasztásáról, melyek jó része onnan ered, hogy a hálózat csomópontjai, az AS-k nem fedik fel szabályrendszerüket. Ezen kérdésekre tudnánk válaszolni úgy, hogy nem a pontos szabályokat adjuk meg, hanem megmutatjuk, hogy adott policy primitívek felhasználásával majdnem ugyanolyan utakat kapunk, mint egy adott AS. Ez azért is lenne egy jó megoldás, mert lehet, hogy egy AS viszonylag bonyolult, sok erőforrást felemésztő szabályokat alkalmaz, pedig nagyon hasonló eredményeket érhetne el úgy is, ha az itt bemutatott p. primitíveket használná.\\

\Aref{modell}. fejezetben a számítógépes hálózatok policy-felderítésével foglalkozó szakirodalom áttekintés után egy általános hálózati- és routing modellt írok le, amely képes kezelni más tudományterületekről származó hasonló problémákat is. Az útvonalválasztás szabályrendszerét egy jól definiált matematikai struktúrával kell meghatározni, hogy definiálni lehessen policy-k közti műveleteket, amelyekkel össze is tudunk kapcsolni policy-ket és tudjuk vizsgálni az kölcsönhatásukat.

\Aref{examples}. fejezetben változatos problémákat és a hozzájuk tartozó routing policy-ket definiálok és leírom a hozzájuk tartozó matematikai struktúrákat.

\Aref{test}. fejezetben szimulációs módszerekkel elemzem, hogy egyes teszt-hálózatokban milyen utakat határoznak meg \aref{modell}. fejezetben definiált policy-k, és javaslatot teszek, hogy egy valós hálózat útvonalválasztását melyik policy-k keverékével lehet legpontosabban modellezni. Ezután több valós hálózatot vizsgálok meg az addigra meghatározott saját policy-k segítségével, és megvizsgálom a keverék policy pontosságát, azaz azt, hogy mennyire térnének el a valós hálózatbeli utak a jelenlegitől akkor, ha az általam ismertetett policy-kel határoznánk meg azokat.
\newline
\newline
A feladatok elvégzéséhez a NetLogo\footnote{\url{http://ccl.northwestern.edu/netlogo/}} nevű hálózati szimulátort fogom használni, a valós hálózatok a BGP hálózat és egy repülési útvonalakat tartalmazó hálózat lesz.

%A Diplomatervben törekszem a szakmai közönség által érthető megfogalmazásra, általában a középiskolai matematikát meg nem haladó eszközrendszerrel dolgozom, az ennél komolyabb fogalmak, definíciók, tételek pedig megtalálhatók a függelékben. Mindazonáltal a számításelmélet, az algoritmuselmélet és egyéb magasabb szintű témakörök ismerete elengedhetetlen ennek a témának a tárgyalása során.

A számításelmélet, az algoritmuselmélet és egyéb magasabb szintű témakörök ismerete elengedhetetlen ennek a diplomamunkának a megértése során, ugyanakkor minden fontosabb matematikai tétel megtalálható a \hyperlink{appendix}{Függelék}ben.

%??????????????? fedését (), illetve a pontosságát

%A diplomaterv célja olyan szabályalapú útvonalépítési stratégiák kutatása (szintetizálása/elemzése), mely mindamellett, hogy alkalmas lehet valós kommunikációs hálózati alkalmazásra is, felhasználható más természetes vagy mesterséges/technológiai valós hálózatok működésének felderítésére, alapvető mozgatórugóinak megismerésére.


% ???
%Tekinthetünk erre a problémára úgy is, mint a számítástechnikában használt \emph{fekete doboz} modellre, amikor tudjuk, hogy mi a rendszer bemenete (), ismerjük a kimenetet, de a pontos működést nem. Természetesen
% ???


%Szabály alapú útvonalépítési stratégiák nagy hálózatokban
%Az Internet rohamos fejlődésével az utóbbi időben mind nagyobb hangsúlyt kapnak a hagyományostól eltérő hálózatmenedzsment funkciókat megvalósító algoritmus kutatások. Ezek egyik legaktívabban művelt ága az útvonalválasztás kérdéseivel foglalkozik, és lényegében az erőforrások optimalizálást célzó klasszikus (főként legrövidebb utak megtalálására koncentráló) algoritmusok alternatíváinak kutatását tűzi ki célul. Az algoritmusok általános szabályok (policy) mentén alakítják ki a lehetséges kommunikációs vonalakat, a kapcsolódó irányadó paraméterek igen változatosak lehetnek. A terület ugyanakkor nem önálló diszciplínaként jelentkezik, lehetőség adódik számos az általános hálózatkutatásban kidolgozott eredmény felhasználására, ill. viszontalkalmazására is.
%A diplomaterv célja olyan szabályalapú útvonalépítési stratégiák kutatása (szintetizálása/elemzése), mely mindamellett, hogy alkalmas lehet valós kommunikációs hálózati alkalmazásra is, felhasználható más természetes vagy mesterséges/technológiai valós hálózatok működésének felderítésére, alapvető mozgatórugóinak megismerésére.

%Szakirodalom alapján tekintse át a modern hálózattudomány legfontosabb eredményeit! Az áttekintés során koncentráljon a hálózati folyamatokra, ezen belül is első sorban a navigálással kapcsolatos eredményekre!
%Alkalmasan választott nagyméretű valós hálózat tanulmányozásán keresztül, elemezze azokat a módszereket, melyek a hálózatbeli általános értelemben vett kommunikációs folyamatokat alapvetően meghatározzák! Állítson össze egy olyan lehetőség szerint minimális primitív szabályrendszer halmazt, mely segítségével a működés jól modellezhető!
%Az eredmények alapján specifikáljon és dolgozzon ki egy olyan modellezési keretrendszert, melynek segítségével nagyméretű valós hálózatokat jellemző útvonalkialakítási dinamikát képes strukturált módon jellemezni!
%A tesztelés során mutassa be keretrendszer alkalmazását hálózati működést definiáló adathalmaz(ok)on!
%Eredményeit részletesen dokumentálja!

%A bevezető tartalmazza a diplomaterv-kiírás elemzését, történelmi előzményeit, a feladat indokoltságát (a motiváció leírását), az eddigi megoldásokat, és ennek tükrében a hallgató megoldásának összefoglalását. A bevezető szokás szerint a diplomaterv felépítésével záródik, azaz annak rövid leírásával, hogy melyik fejezet mivel foglalkozik.
