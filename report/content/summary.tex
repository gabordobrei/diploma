%----------------------------------------------------------------------------
\chapter{Összefoglalás}\label{summary}
%----------------------------------------------------------------------------

A munka első felét elvégeztem, áttekintettem a hálózatkutatás eredményeit, különös tekintettel az útvonalválasztás matematikai kérdésére.\\

\Aref{modell}. fejezetben áttekintettem a szakirodalmat, összeszedve a legfontosabb állomásokat. Rámutattam, hogy az Internet AS-szintű topológiáján kívül, a más tudományterületekről származó problémák útvonalválasztásáról sem tudunk sok mindent és ezért szükséges egy olyan eszköz, ami a policy-feltárás feladatát - általános esetben is - hatékonyan el tudja látni. Ehhez definiáltam a \emph{routing algebrát}, bemutattam a legfontosabb tulajdonságait és műveleteit, emellett a legszélesebb körben használt policy-k algebráit is ismertettem.\\

\Aref{examples}. fejezetben megvizsgáltam a hálózatkutatás szempontjából alapvető modelleket, melyeket a lokális- vagy globális optimalizálás és a közös- vagy egyéni érdekek követése tulajdonságok alapján karakterizáltam. Bemutattam a fertőző betegségek vizsgálatára használt matematikai modellt, megvizsgáltam a témakör útvonalválasztási kérdéseit és kijelöltem a két, a problémakört jól jellemző policy-t. Emellett rávilágított a trend- és a vírusterjedés útvonalválasztási szempontbeli hasonlóságaira és különbségeire és definiáltam két új trendterjedési policy-t.\newline
Megvizsgáltam a már ismertetett policy-kon (ld. \aref{section_algebrapeldak}. alfejezet) kívül az Internet tartományszintű gráfjának az alapszabályát, a völgymentességet, illetve felvázoltam a hiperbolikus térbe ágyazás - általa pedig az elakadásmentes mohó útvonalválasztás - lehetőségét.\\

\Aref{test}. fejezetben bemutattam a vizsgálandó valós hálózatot. Az adatok feldolgozása után meghatároztam a vizsgálandó hálózatot. Definiáltam a vizsgálandó metrikákat, amik mentén össze tudom hasonlítani a valós- és a policy vezérelt útvonalválasztás által meghatározott útvonalakat és hálózatokat. Minden pontpárra megvizsgálom az eredeti és a szimulált út különbségeit (távolság, lépésszám). A valós és a szimulált hálózatok statisztikai összehasonlítását is elvégzem, melyben figyelmet fordítok a fokszámeloszlás, a hálózatok átmérőjének és a pont- ill. él-összefüggőségének összehasonlítására.