%----------------------------------------------------------------------------
\chapter{Összefoglalás}\label{summary}
%----------------------------------------------------------------------------
A feladatkiírás pontjainak eleget tettem. A szakirodalom alapján a hálózatkutatás eredményeit áttekintettem, különös tekintettel az útvonalválasztás matematikai kérdésére. Megvizsgáltam több különböző, valós életből származó hálózati útvonalválasztással szoros kapcsolatban álló problémakört, és mindegyik ilyenhez definiáltam útvonalválasztási stratégiákat, melyek vizsgálata új megvilágításba helyezné ezeket a problémákat. Specifikáltam egy szimulációs keretrendszert, mellyel lehetőség nyílik az új stratégiák vizsgálatára. Egy ilyen vizsgálatot el is végeztem, amely során a repülőtársaságok útvonalválasztását vizsgáltam meg.\\

\Aref{modell}. fejezetben áttekintettem a szakirodalmat, összeszedve a legfontosabb állomásokat. Rámutattam, hogy az Internet AS-szintű topológiáján kívül, a más tudományterületekről származó problémák útvonalválasztásáról sem tudunk sok mindent. Ezért szükséges egy olyan eszköz, ami a policy-feltárás feladatát -- általános esetben is -- hatékonyan el tudja látni. Ehhez definiáltam a \textit{routing algebrákat}, bemutattam a legfontosabb tulajdonságait és műveleteit, emellett a legszélesebb körben használt policy-k algebráit is ismertettem.\\

\Aref{examples}. fejezetben megvizsgáltam és legfontosabb tulajdonságaik alapján karakterizáltam a hálózatkutatás szempontjából alapvető modelleket. Bemutattam a fertőző betegségek vizsgálatára használt matematikai modellt, megvizsgáltam a témakör útvonalválasztási kérdéseit és kijelöltem a két, a problémakört jól jellemző policy-t. Emellett rávilágított a trend- és a vírusterjedés útvonalválasztási szempontbeli hasonlóságaira és különbségeire is és definiáltam két új trendterjedési policy-t. Megvizsgáltam az Internet tartományszintű gráfjának a szabályrendszerét, a völgymentességet, illetve felvázoltam a hiperbolikus térbe ágyazás -- általa pedig az elakadásmentes mohó útvonalválasztás -- lehetőségét.\\

\Aref{framework}. fejezetben specifikáltam és megterveztem egy szimulációs keretrendszert, mellyel valós hálózati problémákon lehet tesztelni különböző útvonalválasztási stratégiák alapján leírt algebrákat. Definiáltam a szimulációhoz szükséges előfeldolgozási, adattisztítási lépéseket és magát a szimulációs folyamatot. Emellett kidolgoztam egy pontrendszert, mellyel értékelni lehet a szimulációs eredményeket. Referencia szimulációkkal úgy alakítottam ki ezt a pontrendszert, hogy ne csak egy relatív skálát kapjak, ahol csak a vizsgált algebrákat hasonlíthatom össze egymással, hanem egy abszolút mércét is jelentsen az eredmény pontértéke.\\

\Aref{test}. fejezetben bemutattam a vizsgálandó valós hálózatot. Az adatok feldolgozása után meghatároztam a vizsgálandó hálózatot. Definiáltam a vizsgálandó metrikákat, amik mentén össze tudtam hasonlítani a valós- és a policy vezérelt útvonalválasztás által meghatározott útvonalakat és hálózatokat. Minden pontpárra megvizsgálta, az eredeti és a szimulált út különbségeit (távolság, lépésszám). A valós és a szimulált hálózatok statisztikai összehasonlítását is elvégeztem, melyben figyelmet fordítottam a fokszámeloszlás, a hálózatok átmérőjének és az élösszefüggőségének összehasonlítására.\\

A saját magam írt szoftverrel futtatott szimulációk eredményeinek alapján kijelenthető, hogy a diplomamunkámban tárgyalt keretrendszer -- megfelelő körülmények között -- alkalmas valós hálózati útvonalválasztási problémák vizsgálatára, algebrák tesztelésére.
