%----------------------------------------------------------------------------
%  A pdf-ben külön oldalon lesznek
%----------------------------------------------------------------------------
% Abstract in hungarian
%----------------------------------------------------------------------------
\chapter*{Kivonat}\addcontentsline{toc}{chapter}{Kivonat}

Az Internet rohamos fejlődésével az utóbbi időben mind nagyobb hangsúlyt kapnak a hagyományostól eltérő hálózatmenedzsment funkciókat megvalósító algoritmus kutatások. Ezek egyik legaktívabban művelt ága az útvonalválasztás kérdéseivel foglalkozik, és lényegében az erőforrások optimalizálást célzó klasszikus (főként legrövidebb utak megtalálására koncentráló) algoritmusok alternatíváinak kutatását tűzi ki célul. Az algoritmusok általános szabályok (policy) mentén alakítják ki a lehetséges kommunikációs útvonalakat.\\

A Diplomaterv célja olyan szabályalapú útvonalépítési stratégiák kutatása (szintetizálása / elemzése), mely mindamellett, hogy alkalmas lehet valós kommunikációs hálózati alkalmazásra is, felhasználható más természetes vagy mesterséges / technológiai valós hálózatok működésének felderítésére.\\

A modern hálózatkutatás még nem adott megoldást arra a problémára, hogy egy hálózat útvonalválasztási szabályrendszerét hogyan határozhatjuk meg pusztán a kialakított utak megfigyelésével. Az útvonalválasztás matematikai modellezése \cite{Compact_Policy_Routing} megfelelő alapot ad egy olyan szimulációs keretrendszer kifejlesztéséhez, amivel erre a problémára egy hiánypótló megoldást adhatunk.\\

Ebben a Diplomatervben leírok egy olyan szimulációs keretrendszert, amely képes a nagyméretű valós hálózatok útvonalválasztási szabályrendszerét vizsgálni és alkalmasan választott új szabályokkal akár javítani is, valamint egy konkrét hálózatot meg is vizsgálok a keretrendszerrel: a repülőtársaságok útvonalválasztási szabályrendszerét modellezem.

\vfill

%----------------------------------------------------------------------------
% Abstract in english
%----------------------------------------------------------------------------

%--- block comment ---
\iffalse


\chapter*{Abstract}\addcontentsline{toc}{chapter}{Abstract}

This document is a \LaTeX-based skeleton for BSc/MSc~theses of students at the Electrical Engineering and Informatics Faculty, Budapest University of Technology and Economics. The usage of this skeleton is optional. It has been tested with the \emph{TeXLive} \TeX~implementation, and it requires the PDF-\LaTeX~compiler.
\vfill

\fi
%--- block comment ---