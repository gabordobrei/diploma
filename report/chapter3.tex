%----------------------------------------------------------------------------
\chapter{Val�s h�l�zatok vizsg�lata}\label{sect:chapter_test}
%----------------------------------------------------------------------------

 % \begin{figure}[!ht]
 % \centering
 % \includegraphics[width=67mm, keepaspectratio]{figures/TeXnicCenter.png}\hspace{1cm}
 % \includegraphics[width=67mm, keepaspectratio]{figures/TeXnicCenter.png}\\\vspace{5mm}
 % \includegraphics[width=67mm, keepaspectratio]{figures/TeXnicCenter.png}\hspace{1cm}
 % \includegraphics[width=67mm, keepaspectratio]{figures/TeXnicCenter.png}
 % \caption{T�bb k�pf�jl beilleszt�se eset�n t�rk�z�ket is �rdemes haszn�lni.} 
 % \label{fig:HVSpaces}
 % \end{figure}

 % \tabref{TabularExample}~t�bl�zat
 % \begin{table}[ht]
	 % \footnotesize
	 % \centering
	 % \caption{Az �rajel-gener�tor chip �rajel-kimenetei.} \label{tab:SysClocks}
	 % \begin{tabular}{ | l | c | c |}
	 % \hline
	 % �rajel & Frekvencia & C�l pin \\ \hline
	 % CLKA & 100 MHz & FPGA CLK0\\
	 % CLKB & 48 MHz  & FPGA CLK1\\
	 % CLKC & 20 MHz  & Processzor\\
	 % CLKD & 25 MHz  & Ethernet chip \\
	 % CLKE & 72 MHz  & FPGA CLK2\\
	 % XBUF & 20 MHz  & FPGA CLK3\\
	 % \hline
	 % \end{tabular}
	 % \label{tab:TabularExample}
% \end{table}

 % \begin{align}
 % \dot{\mathbf{x}}&=\mathbf{A}\mathbf{x}+\mathbf{B}\mathbf{u},\nonumber\\
 % \mathbf{y}&=\mathbf{C}\mathbf{x}\nonumber.
 % \end{align}

 % \begin{align}
 % \begin{bmatrix}
 % a_{11} & a_{12} & \dots & a_{1n}\\
 % a_{21} & a_{22} & \dots & a_{2n}\\
 % \vdots & \vdots & \ddots & \vdots\\
 % a_{m1} & a_{m2} & \dots & a_{mn}
 % \end{bmatrix}
 % \begin{pmatrix}x_1\\x_2\\\vdots\\x_n\end{pmatrix}=
 % \begin{pmatrix}b_1\\b_2\\\vdots\\b_m\end{pmatrix}.
 % \end{align}

 % \begin{align}
 % W(s)=\frac{A}{1+2T\xi s+s^2T^2}.
 % \end{align}

% \begin{lstlisting}[frame=single,float=!ht,caption=P�lda sz�veges irodalomjegyz�k-adatb�zisra BiBTeX haszn�lata eset�n., label=listing:Bibtex]
% asdfasdf
% asdf
% asdf
% afsdf
% \end{lstlisting}

% A diplomatervsablon (a kari ir�nyelvek szerint) az al�bbi f� fejezetekb�l �ll:
 % \begin{enumerate}
	 % \item 1 oldalas \emph{t�j�koztat�} a szakdolgozat/diplomaterv szerkezet�r�l (\verb+guideline.tex+), ami a v�gs� dolgozatb�l t�rlend�,
 % \end{enumerate}