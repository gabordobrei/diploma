\documentclass[11pt,a4paper,oneside]{report}             % Single-side
%\documentclass[11pt,a4paper,twoside,openright]{report}  % Duplex

%\PassOptionsToPackage{chapternumber=Huordinal}{magyar.ldf}
\usepackage{t1enc}
\usepackage[utf8]{inputenc}
\usepackage{amsmath}
\usepackage{algorithm}
\usepackage[noend]{algpseudocode}
\usepackage{amssymb}
\usepackage{enumerate}
\usepackage[thmmarks]{ntheorem}
\usepackage{graphics}
\usepackage{epsfig}
\usepackage{listings}
\usepackage{color}
%\usepackage{fancyhdr}
\usepackage{courier}
\usepackage{lastpage}
\usepackage{anysize}
\usepackage[magyar]{babel}
\usepackage{sectsty}
\usepackage{setspace}  % Ettol a tablazatok, abrak, labjegyzetek maradnak 1-es sorkozzel!
\usepackage[hang]{caption}
\usepackage{hyperref}
\usepackage{xcolor}
\usepackage{cite}
\usepackage{subcaption}
\usepackage{tikz, tikz-3dplot, pgfplots}
\pgfplotsset{compat=newest}
\usepackage{multirow}


%--------------------------------------------------------------------------------------
% Main variables
%--------------------------------------------------------------------------------------
\newcommand{\vikszerzo}{Döbrei Gábor}
\newcommand{\vikkonzulens}{dr. Heszberger Zalán}
\newcommand{\vikcim}{Szabály alapú útvonalépítési stratégiák nagy hálózatokban}
\newcommand{\viktanszek}{Távközlési és Médiainformatikai Tanszék}
\newcommand{\vikdoktipus}{Diplomaterv}
\newcommand{\vikdepartmentr}{Döbrei Gábor}

%--------------------------------------------------------------------------------------
% Page layout setup
%--------------------------------------------------------------------------------------
% we need to redefine the pagestyle plain
% another possibility is to use the body of this command without \fancypagestyle
% and use \pagestyle{fancy} but in that case the special pages
% (like the ToC, the References, and the Chapter pages)remain in plane style

\pagestyle{plain}
%\setlength{\parindent}{0pt} % áttekinthetőbb, angol nyelvű dokumentumokban jellemző
%\setlength{\parskip}{8pt plus 3pt minus 3pt} % áttekinthetőbb, angol nyelvű dokumentumokban jellemző
\setlength{\parindent}{12pt} % magyar nyelvű dokumentumokban jellemző
\setlength{\parskip}{0pt}    % magyar nyelvű dokumentumokban jellemző

\marginsize{35mm}{25mm}{15mm}{15mm} % anysize package
\setcounter{secnumdepth}{0}
\sectionfont{\large\upshape\bfseries}
\setcounter{secnumdepth}{2}
\singlespacing
\frenchspacing

%--------------------------------------------------------------------------------------
%  Setup hyperref package
%--------------------------------------------------------------------------------------
\hypersetup{
    bookmarks=true,            % show bookmarks bar?
    unicode=false,             % non-Latin characters in Acrobat’s bookmarks
    pdftitle={\vikcim},        % title
    pdfauthor={\vikszerzo},    % author
    pdfsubject={\vikdoktipus}, % subject of the document
    pdfcreator={\vikszerzo},   % creator of the document
    pdfproducer={Producer},    % producer of the document
    pdfkeywords={keywords},    % list of keywords
    pdfnewwindow=true,         % links in new window
    colorlinks=true,           % false: boxed links; true: colored links
    linkcolor=black,           % color of internal links
    citecolor=black,           % color of links to bibliography
    filecolor=black,           % color of file links
    urlcolor=black             % color of external links
}

%--------------------------------------------------------------------------------------
% Set up listings
%--------------------------------------------------------------------------------------
\lstset{
  basicstyle=\scriptsize\ttfamily, % print whole listing small
  keywordstyle=\color{black}\bfseries\underbar, % underlined bold black keywords
  identifierstyle=,           % nothing happens
  commentstyle=\color{white}, % white comments
  stringstyle=\scriptsize\sffamily,       % typewriter type for strings
  showstringspaces=false,     % no special string spaces
  aboveskip=3pt,
  belowskip=3pt,
  columns=fixed,
  backgroundcolor=\color{lightgray},
}     
\def\lstlistingname{lista}  

%--------------------------------------------------------------------------------------
%  Some new commands and declarations
%--------------------------------------------------------------------------------------
\newcommand{\code}[1]{{\upshape\ttfamily\scriptsize\indent #1}}

% define references
\newcommand{\figref}[1]{\ref{fig:#1}.}
\renewcommand{\eqref}[1]{(\ref{eq:#1})}
\newcommand{\listref}[1]{\ref{listing:#1}.}
\newcommand{\tabref}[1]{\ref{tab:#1}.}

\newcommand\TODO[1]{\textcolor{red}{ TODO #1}}
\newcommand\todo[0]{\textcolor{red}{ TODO \newline}}

\makeatletter
\def\BState{\State\hskip-\ALG@thistlm}
\makeatother

\makeatletter
\renewcommand*{\ALG@name}{Algoritmus}
\makeatother

\pgfplotsset{
  /pgfplots/colormap={coldredux}{
    [1cm]
    rgb255(0cm)=(255,255,255)
    rgb255(2cm)=(0,192,255)
    rgb255(4cm)=(0,0,255)
    rgb255(6cm)=(0,0,0)
  }
}

\newcommand{\Histo}[3]{
  \begin{figure}[tbh]
    \centering
      \begin{subfigure}[b]{0.49\textwidth}
      \centering
      \resizebox {\textwidth} {!} {
        \begin{tikzpicture}
          \begin{axis}[
              xlabel={A csomópont befoka},
              ylabel={A csomópont kifoka},
              enlarge x limits=0.02,
              enlarge y limits=0.02,
              colorbar,
              colorbar style={
                %ylabel=A csomópontok aránya,
                yticklabel style={
                  text width=2.5em,
                  align=right,
                  /pgf/number format/.cd,
                  fixed,
                  %fixed zerofill
                }
              }
            ]
            \addplot[scatter, scatter src=explicit, only marks, mark=square*]
            file{sim/#1.csv};
          \end{axis}
        \end{tikzpicture}
      }
      \caption{10-es csoportosítás}
    \end{subfigure} \begin{subfigure}[b]{0.49\textwidth}
      \centering
      \resizebox {\textwidth} {!} {
        \begin{tikzpicture}
          \begin{axis}[
              xlabel={A csomópont befoka},
              ylabel={A csomópont kifoka},
              enlarge x limits=0.02,
              enlarge y limits=0.02,
              colorbar,
              colorbar style={
                %ylabel=A csomópontok aránya,
                yticklabel style={
                  text width=2.5em,
                  align=right,
                  /pgf/number format/.cd,
                  fixed,
                  fixed zerofill
                  %{0, 0, , 0.1, , 0.2},
                }
              }
            ]
            \addplot[scatter, scatter src=explicit, only marks, mark=square*]
            file{sim/#1-zoom.csv};
          \end{axis}
        \end{tikzpicture}
      }
      \caption{A maximum 30 ki- és befokú csúcsok.}
    \end{subfigure}
    \caption{#2 \label{#3}}
  \end{figure}
}

\newcommand{\AvgPlot}[3]{
  \begin{figure}[tbh]
    \centering
      \begin{subfigure}[b]{0.49\textwidth}
      \centering
      \resizebox {\textwidth} {!} {
        \begin{tikzpicture}
          \begin{axis}[
            xlabel={A csomópont átlagfokszáma},
            ylabel={A csomópontok aránya},
            enlarge x limits=0.02,
            enlarge y limits=0.02,
            legend pos=north west
          ]
          \addplot[smooth,color=blue]
          file{sim/#1-avg.csv};
          \end{axis}
        \end{tikzpicture}
      }
      \caption{Az átlagfokszám-eloszlás.}
    \end{subfigure} \begin{subfigure}[b]{0.49\textwidth}
      \centering
      \resizebox {\textwidth} {!} {
        \begin{tikzpicture}
          \begin{axis}[
            xlabel={A csomópont átlagfokszáma},
            ylabel={A csomópontok aránya},
            enlarge x limits=0.02,
            enlarge y limits=0.02,
            yticklabels={0, 0, , 0.1, , 0.2},
            legend pos=north west
          ]
          \addplot[smooth,color=blue]
          file{sim/#1-avg-zoom.csv};
          \end{axis}
        \end{tikzpicture}
      }
      \caption{A maximum 50 átlagfokszámú csomópontok.}
    \end{subfigure}
    \caption{#2 \label{#3}}
  \end{figure}
}


\DeclareMathOperator*{\argmax}{arg\,max}
%\DeclareMathOperator*[1]{\floor}{arg\,max}
\DeclareMathOperator{\sign}{sgn}
\DeclareMathOperator{\rot}{rot}
\definecolor{lightgray}{rgb}{0.95,0.95,0.95}

\author{\vikszerzo}
\title{\viktitle}
\includeonly{
  titlepage,%
  declaration,%
  acknowledgement,%
  abstract,%
  content/introduction,%
  content/chapter1,%
  content/chapter2,%
  content/chapter3,%
  content/chapter4,%
  content/summary,%
  bibliography,
  content/appendices,%
}
%--------------------------------------------------------------------------------------
%  Setup captions
%--------------------------------------------------------------------------------------
\captionsetup[figure]{
%labelsep=none,
%font={footnotesize,it},
%justification=justified,
width=.75\textwidth,
aboveskip=10pt}

\renewcommand{\captionlabelfont}{\small\bf}
\renewcommand{\captionfont}{\footnotesize\it}

%--------------------------------------------------------------------------------------
% Table of contents and the main text
%--------------------------------------------------------------------------------------
\begin{document}
\pagenumbering{arabic}
\onehalfspacing
\include{titlepage}
\include{declaration}
%----------------------------------------------------------------------------
\chapter*{Köszönetnyilvánítás}\addcontentsline{toc}{chapter}{Köszönetnyilvánítás}
%----------------------------------------------------------------------------

Nagyon sokat köszönhetek \vikkonzulens nak, aki alapdiplomás szakdolgozati, majd mesterdiplomás konzulensként nem csak elindított a kutatási irányba, de menet közben sok konzultációval, és baráti vezetéssel segítette fejlődésemet. A közös munka inspirációt is jelentett és megismertetett sok olyan eredménnyel, cikkel, amelyre egymagam nem találtam volna rá.\\*

Sok közvetett segítséget és érdeklődést kaptam barátaimtól, évfolyamtársaimtól. Végül családom, elsősorban feleségem végtelen türelmére szükség volt a dolgozat létrejöttéhez.

\tableofcontents\vfill
%----------------------------------------------------------------------------
\chapter*{Kivonat}\addcontentsline{toc}{chapter}{Kivonat}
%----------------------------------------------------------------------------
Az Internet rohamos fejlődésével az utóbbi időben mind nagyobb hangsúlyt kap a hagyományostól eltérő hálózatmenedzsment funkciókat megvalósító algoritmusok kutatása. Ezek egyik legaktívabban művelt ága az útvonalválasztás (routing) kérdéseivel foglalkozik, és lényegében az erőforrások optimalizálást célzó klasszikus (főként legrövidebb utak megtalálására koncentráló) algoritmusok alternatíváinak kutatását tűzi ki célul.\\

A kommunikációs hálózatok általános szabályok (policy) mentén alakítják ki a lehetséges útvonalakat. A modern hálózatkutatás még nem adott megoldást arra a problémára, hogy hogyan deríthetjük fel egy hálózat szabályrendszerét a kialakított utak megfigyelésével. Az útvonalválasztási szabályok matematikai modellezése megfelelő alapot ad\cite{Compact_Policy_Routing} egy olyan rendszer kifejlesztéséhez, amivel erre a problémára egy hiánypótló megoldást adhatunk.\\

Ebben a Diplomatervben leírok egy olyan szimulációs keretrendszert, amely képes a nagyméretű valós hálózatok útvonalválasztási szabályrendszerét vizsgálni és alkalmasan választott új szabályokkal akár javítani is. A szimulátor működését be is mutatom egy konkrét hálózat elemzésével: a repülőtársaságok útvonalválasztási szabályrendszerét azonosítom.

\vfill

%----------------------------------------------------------------------------
\chapter*{Abstract}\addcontentsline{toc}{chapter}{Abstract}
%----------------------------------------------------------------------------
Nowadays, a lot of researches are dealing with algorithms that implements alternative network management functions. One of the most popular branches of these researches are about routing and more particularly about optimizing classical algorithms -- mostly shortest path finding algorithms -- in terms of used resources.\\

In communication networks, routing decisions are based on routing policies. There is no solution for the problem when we want to determine the policies by the observation of the network. However by using the mathematical models of routing policies, we will be able to create a simulator which helps discover this policies.\\

In this thesis my goal is to build a simulation framework that can determine the routing policies of real large-scale networks. The simulation framework uses the mathematical models of the routing policies. I will define a measuring system to compare the simulation results. I will use the simulator to examine the policies behind the route-defining decisions of the largest airlines.

\vfill
%----------------------------------------------------------------------------
\chapter*{Bevezet�}\addcontentsline{toc}{chapter}{Bevezet�}
%----------------------------------------------------------------------------

Az Internet rohamos fejl�d�s�vel egyre nagyobb ig�ny mutatkozik a komplex sz�m�t�g�pes h�l�zatok megismer�s�re. �ltal�nos esetben a (kis) h�l�zatok strukt�r�ja �s m�k�d�si mechanizmusai j�l meghat�rozottak, hiszen a saj�t tervez�s�nk eredm�nyek�nt j�ttek l�tre �s a vez�rl�s is a mi kez�nkben van. Pontosan tudjuk, hogy egy h�l�zati csom�pont melyik m�sik csom�ponttal van kapcsolatban, ismerj�k az �sszek�ttet�seket �s az �tvonalv�laszt�st meghat�roz� szab�lyokat is. Ha b�rmilyen m�dos�t�st szeretn�nk eszk�z�lni, vagy egy hib�t szeretn�nk kijav�tani, azonnal (nagyon gyorsan) tudjuk, hogy hol kell beavatkozni.
% ?? kell ENTER?
Azonban egy olyan h�l�zat vizsg�lata sor�n, amelyet nem mi tervezet�nk, nagyon gyorsan szembes�l�nk olyan k�rd�sekkel, amiket csak nehezen �s sok munka �r�n tudunk megv�laszolni - m�r�sekkel, tesztel�ssel - �s csak abban az esetben, ha a h�l�zat m�rete m�g nem jelent probl�m�t.\\

Az �tvonalakat meghat�roz� algoritmusok �ltal�nos szab�lyok (policy) ment�n alak�tj�k ki a lehets�ges (kommunik�ci�s) �tvonalakat �s b�r a kapcsol�d� ir�nyad� param�terek igen v�ltozatosak lehetnek, a probl�mak�r egy meghat�roz� param�tere szerinti legr�videbb �tvonalat szoktuk a legjobbnak tekinteni. M�gis, ha megvizsg�ljuk az Internet magas szint� topol�gi�j�t\footnote{Nevezik m�g az Internet tartom�ny-szint�-, vagy AS-szint� topol�gi�j�nak, gr�fj�nak is.} �s a benne kialakult utakat, akkor azt vessz�k �szre, hogy ezek az utak nem az optim�lis megold�sok, legal�bbis nem a legr�videbbek. Mivel feltehetj�k, hogy az anyagi haszon maximaliz�l�sa c�lj�b�l az Auton�m Rendszerek\footnote{Auton�m Rendszer - Autonomous System (AS)- : �n�ll� �tv�laszt�si tartom�ny, amelyen bel�l egyetlen, j�l meghat�rozott �tvonalv�laszt�si szab�ly �rv�nyes�l.} �zemeltet�i racion�lis d�nt�sek r�v�n �p�tett�k ki pontosan ezeket az utakat, jogos k�rd�s, hogy pontosan milyen strat�gia alapj�n tett�k ezt. Mi vezethet egy l�tsz�lag �sszer�tlen d�nt�shez? �ltal�noss�gban igaz az egy�b h�l�zatokra is, hogy az adott szitu�ci� optim�lis �tjai nem t�nnek racion�lisnak.\\

Ha k�rben�z�nk a vil�gban, az informatik�t�l t�voles� ter�leteken is rengeteg p�ld�t tal�lunk olyan h�l�zatokra, amiknek nem �rtj�k m�g a m�k�d�s�t, nem tudjuk pontosan le�rni a bels� folyamatait. Sz�mos p�ld�t tal�lunk a biol�gi�b�l, a szociol�gi�b�l vagy a p�nz�gyi vil�gb�l, ugyanakkor minden ilyen probl�ma vizsg�lat�t vissza lehet vezetni egy olyan modell vizsg�lat�ra, ami �ltal�nosan k�pes kezelni mag�t a h�l�zat fogalm�t �s az azon t�rt�n� �tvonalak kialakul�s�t/kialak�t�s�t. A v�rusok �ltal terjesztett betegs�g terjed�se, egy ruhaviselet divatt� v�l�sa, �s a sz�m�t�g�pes h�l�zatok kommunik�ci�s �tjainak kialakul�sa modellezhet� a gr�felm�let eszk�zrendszer�vel. Mindh�rom esetben j�l meghat�rozhat�, elk�l�n�l� csom�pontok vannak, akik k�z�tt terjed egy csomag, ami lehet inform�ci� vagy pl. a betegs�g maga. Egy modellen bel�l a csom�pontok �ltal�ban nem k�l�nb�znek egym�st�l, �s a k�zt�k lev� kapcsolatok k�l�nb�z� tulajdons�gokkal rendelkeznek, amit az �tvonalak kialakul�sa k�zben figyelembe is kell venni. Nem t�telez�nk fel k�l�nbs�get k�t ember k�z�tt, b�rki meg tud betegedni. Mag�t�l �rtet�d�nek l�tszik ugyanakkor, hogy leveg�ben terjed� v�rusos fert�z�s j�val laz�bb kapcsolaton kereszt�l is tov�bbterjed, m�g egy v�r �tj�n terjed� betegs�g sokkal szorosabb kapcsolaton tud csak tov�bbterjedni: a kapcsolatokat s�lyozni kell. Ha egy ember fog�konyabb egy betegs�gre, akkor a hozz� tartoz� �leken nagyobb val�sz�n�s�ggel terjed tov�bb majd a betegs�g. Ugyanilyen meggondol�sb�l, k�t szomsz�dos router k�z�tt a nagyobb s�vsz�less�g� �ton tov�bb�tjuk a csomagokat. �rdekes azonban, hogy egy divat elterjed�s�t m�r nem tudjuk ilyen egyszer�en le�rni, hiszen nem is igaz�n az a meghat�roz�, hogy mennyire befoly�sos �s ismert emberek rekl�mozz�k, hanem az, hogy a t�rsadalom felk�sz�lt-e m�r a befogad�sra \cite{DuncanWatts, DobreiMScOnlab1}.\\

A Diplomatervben a v�gs� c�l olyan policy primit�veket meghat�rozni, melyek �sszekapcsol�s�val j�l lehet k�zel�teni a val�s h�l�zatokat. Jelenleg sok megv�laszolatlan k�rd�s van a BGP h�l�zat �tvonalv�laszt�s�r�l, melyek j� r�sze onnan ered, hogy a h�l�zat csom�pontjai, az AS-k nem fedik fel szab�lyrendszer�ket. Ezen k�rd�sekre tudn�nk v�laszolni �gy, hogy nem a pontos szab�lyokat adjuk meg, hanem megmutatjuk, hogy adott policy primit�vek felhaszn�l�s�val majdnem ugyanolyan utakat kapunk, mint egy adott AS. Ez az�rt is lenne egy j� megold�s, mert lehet, hogy egy AS viszonylag bonyolult, sok er�forr�st felem�szt� szab�lyokat alkalmaz, pedig nagyon hasonl� eredm�nyeket �rhetne el �gy is, ha az itt bemutatott p. primit�veket haszn�ln�.\\

\Aref{sect:chapter_modell}. fejezetben a sz�m�t�g�pes h�l�zatok policy-felder�t�s�vel foglalkoz� szakirodalom �ttekint�s ut�n egy �ltal�nos h�l�zati- �s routing modellt �rok le, amely k�pes kezelni m�s tudom�nyter�letekr�l sz�rmaz� hasonl� probl�m�kat is. Az �tvonalv�laszt�s szab�lyrendszer�t egy j�l defini�lt matematikai strukt�r�val kell meghat�rozni, hogy defini�lni lehessen policy-k k�zti m�veleteket, amelyekkel �ssze is tudunk kapcsolni policy-ket �s tudjuk vizsg�lni az k�lcs�nhat�sukat.

\Aref{sect:chapter_examples}. fejezetben v�ltozatos probl�m�kat �s a hozz�juk tartoz� routing policy-ket defini�lok �s le�rom a hozz�juk tartoz� matematikai strukt�r�kat.

\Aref{sect:chapter_test}. fejezetben szimul�ci�s m�dszerekkel elemzem, hogy egyes teszt-h�l�zatokban milyen utakat hat�roznak meg \aref{sect:chapter_modell}. fejezetben defini�lt policy-k, �s javaslatot teszek, hogy egy val�s h�l�zat �tvonalv�laszt�s�t melyik policy-k kever�k�vel lehet legpontosabban modellezni. Ezut�n t�bb val�s h�l�zatot vizsg�lok meg az addigra meghat�rozott saj�t policy-k seg�ts�g�vel, �s megvizsg�lom a kever�k policy pontoss�g�t, azaz azt, hogy mennyire t�rn�nek el a val�s h�l�zatbeli utak a jelenlegit�l akkor, ha az �ltalam ismertetett policy-kel hat�rozn�nk meg azokat.
\newline
\newline
A feladatok elv�gz�s�hez a NetLogo\footnote{\url{http://ccl.northwestern.edu/netlogo/}} nev� h�l�zati szimul�tort fogom haszn�lni, a val�s h�l�zatok a BGP h�l�zat �s egy rep�l�si �tvonalakat tartalmaz� h�l�zat lesz.

%A Diplomatervben t�rekszem a szakmai k�z�ns�g �ltal �rthet� megfogalmaz�sra, �ltal�ban a k�z�piskolai matematik�t meg nem halad� eszk�zrendszerrel dolgozom, az enn�l komolyabb fogalmak, defin�ci�k, t�telek pedig megtal�lhat�k a f�ggel�kben. Mindazon�ltal a sz�m�t�selm�let, az algoritmuselm�let �s egy�b magasabb szint� t�mak�r�k ismerete elengedhetetlen ennek a t�m�nak a t�rgyal�sa sor�n.

A sz�m�t�selm�let, az algoritmuselm�let �s egy�b magasabb szint� t�mak�r�k ismerete elengedhetetlen ennek a diplomamunk�nak a meg�rt�se sor�n, ugyanakkor minden fontosabb matematikai t�tel megtal�lhat� a \hyperlink{appendix}{F�ggel�k}ben.

%??????????????? fed�s�t (), illetve a pontoss�g�t

%A diplomaterv c�lja olyan szab�lyalap� �tvonal�p�t�si strat�gi�k kutat�sa (szintetiz�l�sa/elemz�se), mely mindamellett, hogy alkalmas lehet val�s kommunik�ci�s h�l�zati alkalmaz�sra is, felhaszn�lhat� m�s term�szetes vagy mesters�ges/technol�giai val�s h�l�zatok m�k�d�s�nek felder�t�s�re, alapvet� mozgat�rug�inak megismer�s�re.


% ???
%Tekinthet�nk erre a probl�m�ra �gy is, mint a sz�m�t�stechnik�ban haszn�lt \emph{fekete doboz} modellre, amikor tudjuk, hogy mi a rendszer bemenete (), ismerj�k a kimenetet, de a pontos m�k�d�st nem. Term�szetesen
% ???


%Szab�ly alap� �tvonal�p�t�si strat�gi�k nagy h�l�zatokban
%Az Internet rohamos fejl�d�s�vel az ut�bbi id�ben mind nagyobb hangs�lyt kapnak a hagyom�nyost�l elt�r� h�l�zatmenedzsment funkci�kat megval�s�t� algoritmus kutat�sok. Ezek egyik legakt�vabban m�velt �ga az �tvonalv�laszt�s k�rd�seivel foglalkozik, �s l�nyeg�ben az er�forr�sok optimaliz�l�st c�lz� klasszikus (f�k�nt legr�videbb utak megtal�l�s�ra koncentr�l�) algoritmusok alternat�v�inak kutat�s�t t�zi ki c�lul. Az algoritmusok �ltal�nos szab�lyok (policy) ment�n alak�tj�k ki a lehets�ges kommunik�ci�s vonalakat, a kapcsol�d� ir�nyad� param�terek igen v�ltozatosak lehetnek. A ter�let ugyanakkor nem �n�ll� diszcipl�nak�nt jelentkezik, lehet�s�g ad�dik sz�mos az �ltal�nos h�l�zatkutat�sban kidolgozott eredm�ny felhaszn�l�s�ra, ill. viszontalkalmaz�s�ra is.
%A diplomaterv c�lja olyan szab�lyalap� �tvonal�p�t�si strat�gi�k kutat�sa (szintetiz�l�sa/elemz�se), mely mindamellett, hogy alkalmas lehet val�s kommunik�ci�s h�l�zati alkalmaz�sra is, felhaszn�lhat� m�s term�szetes vagy mesters�ges/technol�giai val�s h�l�zatok m�k�d�s�nek felder�t�s�re, alapvet� mozgat�rug�inak megismer�s�re.

%Szakirodalom alapj�n tekintse �t a modern h�l�zattudom�ny legfontosabb eredm�nyeit! Az �ttekint�s sor�n koncentr�ljon a h�l�zati folyamatokra, ezen bel�l is els� sorban a navig�l�ssal kapcsolatos eredm�nyekre!
%Alkalmasan v�lasztott nagym�ret� val�s h�l�zat tanulm�nyoz�s�n kereszt�l, elemezze azokat a m�dszereket, melyek a h�l�zatbeli �ltal�nos �rtelemben vett kommunik�ci�s folyamatokat alapvet�en meghat�rozz�k! �ll�tson �ssze egy olyan lehet�s�g szerint minim�lis primit�v szab�lyrendszer halmazt, mely seg�ts�g�vel a m�k�d�s j�l modellezhet�!
%Az eredm�nyek alapj�n specifik�ljon �s dolgozzon ki egy olyan modellez�si keretrendszert, melynek seg�ts�g�vel nagym�ret� val�s h�l�zatokat jellemz� �tvonalkialak�t�si dinamik�t k�pes struktur�lt m�don jellemezni!
%A tesztel�s sor�n mutassa be keretrendszer alkalmaz�s�t h�l�zati m�k�d�st defini�l� adathalmaz(ok)on!
%Eredm�nyeit r�szletesen dokument�lja!

%A bevezet� tartalmazza a diplomaterv-ki�r�s elemz�s�t, t�rt�nelmi el�zm�nyeit, a feladat indokolts�g�t (a motiv�ci� le�r�s�t), az eddigi megold�sokat, �s ennek t�kr�ben a hallgat� megold�s�nak �sszefoglal�s�t. A bevezet� szok�s szerint a diplomaterv fel�p�t�s�vel z�r�dik, azaz annak r�vid le�r�s�val, hogy melyik fejezet mivel foglalkozik.
%----------------------------------------------------------------------------
%Onnan indulok, hogy az 1990-es évek végétől egyre többet foglalkoztak a BGP feltárásával, amivel el is jutottak addig, hogy van egy viszonylag pontos kép a hálózatról, de arra, hogy miért egy adott policy-t használnak az AS-ek, nem tudnak válaszolni. Itt hozom fel, hogy van más olyan terület is, amit még nem értünk, ezért is lenne jó egy olyan eszköz, ami ilyesmit tudna csinálni. Ehhez a policy-kat felbontom egyszerű primitívekre, és azoknak a kombinálásával hozok ki komplexebb policy-kat.

%1.2 fejezet: policy-k definíciója,
%1.3 fejezet: műveletek policy-k között
%1.4 fejezet: policy-k tulajdonságai


%----------------------------------------------------------------------------

\newtheorem{definition}{Definíció}
\newtheorem{conjecture}{Sejtés}
\newtheorem{lemma}{Lemma}
\newtheorem{theorem}{Tétel}
\newtheorem{note}{Megjegyzés}

\numberwithin{definition}{section}
\numberwithin{conjecture}{section}
\numberwithin{lemma}{section}
\numberwithin{theorem}{section}
\numberwithin{note}{section}
\numberwithin{table}{section}

%\setcounter{secnumdepth}{3}


%----------------------------------------------------------------------------
\chapter{Általánosított hálózatok és útvonalválasztási szabályok}\label{modell}
%----------------------------------------------------------------------------

Az Internet gerinchálózatát kialakító szabályrendszert nagy vonalaiban ismerjük. Az Internet AS-szintű topológiáját a BGP\footnote{Border Gateway Protocol} határozza meg. Különböző eljárásokat ismerünk az Autonóm Rendszerek közti kapcsolatok feltárására a BGP-s routing táblák alapján, illetve az Internet router-szintű topológiájából, sőt az AS-ek routing-policy\footnote{Útvonalválasztási szabályrendszer}-ját is tudjuk becsülni, ugyanakkor arra a kérdésre eddig még senki sem tudott választ adni, hogy miért éppen így alakultak ki a kommunikációs utak.\\

Ebben a fejezetben a szakirodalom rövid áttekintése után egy általános hálózati- és routing modellt mutatok be. Az útvonalválasztás szabályrendszerét egy jól definiált matematikai struktúrával - a routing algebrákkal - írom le, és definiálom a policy-k közti műveleteket, amelyekkel össze is tudunk kapcsolni policy-kat. Emellett egy alfejezet foglalkozik a policy-k különböző tulajdonságaival, melyek figyelembevétele fontos szempont a szimulációk megtervezésénél.

  %----------------------------------------------------------------------------
  \section{Történeti áttekintés}\label{section_tortenelem}
  %----------------------------------------------------------------------------

  Az 1990-es évek végére a kutatók felismerték a tényt, hogy a BGP szintű Internet hálózatról nem tudunk szinte semmit. Rendelkezésükre állt néhány szolgáltató BGP-s routing táblája, de ez nagyon kevésnek bizonyult, hiszen a legtöbb AS közötti kapcsolat titkos gazdasági döntés révén született, így pontos képük nem lehetett a hálózatról. Ekkor kezdték el vizsgálni a különböző lehetőségeket, hogy hogyan lehetne feltárni ezt a rejtett hálózatot. Nem tűnt reménytelennek a helyzet, hiszen hálózat széle - a végpontok - nem tartoznak a szolgáltatókhoz, így egy tetszőleges számítógépről indított csomag útját végigkövetve értékes információkat nyerhettek. Természetesen a \emph{traceroute}\footnote{Egy számítógép-hálózati diagnosztikai eszköz, amivel az IP (Internet Protokoll) hálózaton haladó csomagok útját lehet követni.} futási eredményeit elemezve egy halom adatot kapunk, amit nehezen lehet csak feldolgozni, annál is inkább, mert az egyes a mérések ismétlése során más és más útvonalon haladt át a követendő csomag. Készültek szimulátorok, amik ügyesen kezelték a nehézségeket, és különböző heurisztikákat használtak, hogy csökkentsék a szükséges mérések számát \cite{Heuristics_for_Internet_Map_Discovery}, ugyanakkor még mindig nagy szakadék volt az eredmények és az elvárások között. Nem volt egyértelmű, hogy egy csomag miért épp az adott AS-eken keresztül ért célba.\\

  Az ezredforduló után már sok új kutatás foglalkozott az Internet AS-szintű topológiájának feltérképezésével úgy, hogy az ehhez szükséges információt a kapcsolatokról még mindig csak a BGP routing táblákból szerezték. Volt azonban egy újfajta megközelítést is, ami az önálló AS-kapcsolatokat tárta fel az Internet router-szintű topológiájából \cite{Inferring_AS_level_Internet_Topology_from_Router_Level_Path_Traces}. Ez több szempontból is előnyös volt, mint a BGP-s routing táblákból kiolvasott topológia, hiszen
  \begin{enumerate}
    \item Nagyobb felbontásban látjuk az AS-szintű térképet (pl. látszódnak a többszörös kapcsolatok AS-ek között);
    \item Látszódnak a BGP protokoll által aggregált - így az AS-szintű hálózatban eltakart - útvonalak;
    \item Így már lehetőség volt azonosítani a határ-routereket\footnote{Azon routereket, amelyek az AS szélén vannak és a többi AS-hez jelentik a kapcsolatot, határ-routernek nevezzük.}, aminek segítségével pontosabban karakterizálhatták az AS-en belül kapcsolatokat.
  \end{enumerate}

  Az addigi eredményeket felhasználva már volt egy viszonylag pontos kép az Internet tartomány-szintű topológiájáról, amit tovább pontosítottak úgy, hogy az egyes AS-eken belüli szabályokat is feltárták, hiszen keveset tudtak még arról, hogy milyen routing policy-t használnak az AS-k. A BGP protokoll ugyanis lehetővé teszi az AS-eknek, hogy megválasszák az útvonalválasztási policy-jukat, ami alapján történik a csomagtovábbítás és az elérhetőségi adatok terjesztése az AS-en belül. Megmutatták, hogy az AS-k a többi szolgáltatók csak egy csoportjának hirdetik magukat, ami mögött valószínűleg traffic engineering\footnote{Forgalomszabályozás: tudatos tervezéssel próbálják meg elkerülni azokat az eseteket, amikor szolgáltatás-kimaradás lép fel a túlterhelés miatt.} lehet. Pl. több Tier-1-es\footnote{A BGP hierarchiában a legmagasabb szintű szolgáltatók, ezeket követi a Tier-2 és a Tier-3.} AS is az ügyfeleit (közvetve vagy direkt) a peer kapcsolatain keresztül éri el a közvetlen customerei helyett. Ezenkívül a válogatott hirdetés szerint jóval kevesebb elérhető útvonal van az Interneten, mint azt az AS kapcsolati gráf mutatja (Ezért is volt jóval pontatlanabb a BGP routing táblák alapján elképzelt kép.) \cite{On_Inferring_and_Characterizing_Internet_Routing_Policies}.

  Miután már volt valamilyen fogalmunk arról, hogy az egyes AS-k hogyan működtetik hálózatukat, az AS-ek közötti kapcsolatok mélyebb megismerése következett. Adva van az Internetnek már megkülönböztetett router-szintű és AS-szintű topológiái, illetve felületesen már ismerjük az egyes policy-kat. Azt kezdték el vizsgálni, hogy a router-szintű topológiában mi lenne a legrövidebb út és ehhez képest mi az adott (valós) policy-út \cite{The_Impact_of_Routing_Policy_on_Internet_Paths}. Több szempontból is vizsgálták a kérdést:
  \begin{itemize}
    \item Mennyivel fújja fel a policy a shortest-utat\footnote{Shortest-út: Azt a policy-t, amely a legrövidebb utakat választja, shortest-path (legrövidebb-út) policy-nak nevezzük.}?
    \item Létezik-e $S$ és $D$ forrás-cél pár között egy $I$ köztes pont, amire a $d_{policy}( S \rightarrow I ) + d_{policy}( I \rightarrow D ) < d_{policy}( S \rightarrow D )$? (Ebben az esetben lehetne egy ilyen közbeiktatott ponttal javítani)
    \item A policy-k vajon a nagyobb AS-k felé terelik a forgalmat?
  \end{itemize}

  Ugyan abban az időben még csak a legrövidebb utakat vizsgálták és azokat is a multicasting\footnote{A multicast egy információtovábbítási mód, amikor egy üzenetet több célhoz szeretnénk eljuttatni egyidejűleg egyetlen forrástól.} szempontjából, de azóta a kutatók felismerték, hogy olyan kérdések várnak még válaszra, amelyek alapvetően befolyásolják az egész Internet gerinchálózatát. Ezenkívül jelentősen megkönnyítené, illetve pontosabbá tehetné a témakörbeli kutatásokat, ha rendelkezésünkre állna egy olyan hálózati modell, amely nem csak a skálafüggetlenséget\footnote{Egy hálózat skálafüggetlen, ha benne a fokszámeloszlás hatványfüggvényt követ. Az Internet hálózata tipikusan skálafüggetlen hálózat. (A kifejezés Barabási Albert László magyar származású amerikai matematikus nevéhez fűződik.)} tudja szimulálni, hanem a - gazdasági érdekek miatt titkolt - kapcsolatokat és útvonalválasztást is valósághűen tudja modellezni.\\

  Közel húsz év óta foglalkoztatja a kutatókat a hálózatok kialakulása és a kialakult útvonalak, ugyanakkor az Internet AS-szintű topológiájának vizsgálatakor arról nem állítanak semmit, hogy az egyes policy-kat miért használják. Sőt, ugyanígy nem tudunk szinte semmit a már említett más területekről származó problémák esetében sem.

  Jelen Diplomaterv keretében egy olyan megoldást mutatok be, amely képes megbecsülni, hogy miért - milyen szabályszerűségek figyelembevételével - hoztak meg egyes döntéseket. Azzal a feltételezéssel élek, hogy minden policy-t fel lehet építeni jól meghatározott, egyszerű primitívekből, illetve ezen primitívek azonosítása és változatos összekapcsolása révén meg fogjuk tudni mondani, hogy egy már kialakult hálózatot milyen komplex policy határoz meg.\\

  %----------------------------------------------------------------------------
  \section{Útvonalválasztás, routing algebrák}\label{section_routingalgebrak}
  %----------------------------------------------------------------------------

  Az általános hálózati modellben egy véges, egyszerű és összefüggő $G(V;E)$ gráfot használunk a hálózat megadására, ahol $|V| = n$ és $|E| = m$. Az élek irányítottsága és súlyozása az adott problémától függ, minden kombináció megengedett. Legyen $deg(v)$ a $v \in V$ csúcs fokszáma és legyen $d = max_{v \in V} deg(v)$. Egy $s-t$ séta a csúcsoknak egy $p = (s =)v_1, v_2, ..., v_k(= t)$ sorozata, ahol $k$ a séta hossza és $(v_i, v_{i+1}) \in E$ : $\forall i = 1,...,k-1$. Ezenkívül egy kör olyan séta, ahol $s = t$.

  Ahhoz, hogy meg tudjuk mondani, hogy két útból melyiket érdemes választani, definiálni kell egy preferencia sorrendet az utak között. Ehhez pedig az utak hosszára lesz szükség, amit egy metrikával, azaz távolságfüggvénnyel tudunk mérni. A metrikus tér fogalma egy (halmaz, függvény) párt jelent, ahol a függvény bármely két, halmazbeli elemhez egy nemnegatív valós számot rendel (vagyis a távolságukat méri).

  \begin{definition}[Metrikus tér]\label{eq:MetrikusTerDef}
    A metrikus tér egy olyan (X, $\delta$) pár, ahol X egy tetszőleges halmaz,\\$\delta$: $X^{2}$ $\rightarrow$ $\mathbb{R}^{+}_{0}$ pedig egy olyan nemnegatív valós függvény, melyre tetszőleges x, y, z $\in$ X esetén:
    \begin{enumerate}
    \item \emph{ $\delta$(x, y) = 0 $\Leftrightarrow$ x = y } (megegyezőségi tulajdonság)
    \item \emph{ $\delta$(x, y) = $\delta$(y, x) } (szimmetria)
    \item \emph{ $\delta$(x, z) $\leq$ $\delta$(x, y) + $\delta$(y, z) } (háromszög-egyenlőtlenség).
    \end{enumerate}
  \end{definition}

  Annak érdekében, hogy  minél általánosabb lehessen a modell, minden $e \in E$ élet egy tulajdonság-vektor jellemez, amely vektor minden dimenziójának értékeit különböző metrikák szerint adhatunk meg. Így definiálhatjuk az él hosszát, (sáv)szélességét, késleltetését, megbízhatóságát, sőt bármilyen nem tipikus tulajdonságot is (pl. szín, vagy egy időtől függő $f(t)$ függvényt, stb.).

  A metrikák fontos szerepet játszanak az útvonalválasztásban, hiszen ez alapján tudjuk a hálózati utak közötti preferenciát megadni, másképp megfogalmazva megadja, hogy egy bizonyos tulajdonság alapján mekkora a költsége egy útnak. Az útvonalválasztás során a metrika lehet statikus, amikor egy előre rögzített elvet követünk végig, vagy lehet dinamikus, amikor a hálózat adott állapotától függően automatikusan változik. Azt, hogy milyen metrika szerint végezzük a routing-ot nevezzük \emph{routing policy}-nak:

  \begin{definition} [Routing policy]
    A routing policy egy olyan $p_{st}^{*}=Pol(\mathcal{P}_{st})$ függvény, aminek az értelmezési tartománya a lehetséges $s - t$ utak: $\mathcal{P}_{st}$ és az adott policy-nak megfelelő legkedvezőbb utat adja vissza.
  \end{definition}

  Ahhoz, hogy ezentúl matematikailag is kényelmesen tudjuk kezelni a policy-kat, az ún. routing algebrák fogalmát kell használnunk, melyek az általános útvonalválasztó policy-k matematikai leírása \cite{Sobrinho_Algebra_and_Algorithms, Compact_Policy_Routing}.

  \begin{definition} [Routing algebra]
    Az $\mathcal{A}$ routing algebra\footnote{Az algebra elnevezés arra utal, hogy - mint később látni fogjuk - műveleteket lehet végezni ezen objektumokon.} egy teljesen rendezett félcsoport egy ,,végtelen elemmel'': $\mathcal{A}~=~(W,~\phi,~\bigoplus,~\preceq)$, ahol $W$ az élek súlyainak halmaza, $\phi~(\phi \notin W)$ egy speciális végtelen súly, abban az értelemben, hogy azon az élen vagy úton nem lehet átmenni, a $\bigoplus$ a súlyok egy kétváltozós kompozíció operátora, a $\preceq$ pedig a súlyok összehasonlító operátora.
  \end{definition}

  Még pontosabban, a következő tulajdonságokat követeljük meg:
  \begin{itemize}
    \item \emph{ ($W,~\bigoplus$) egy Abel-csoport:}
    \begin{itemize}
    \item \emph{ Zárt:} $w_{1} \bigoplus w_{2}~\in W, ~\forall w_{1}, w_{2}\in W$
    \item \emph{ Asszociatív:} $(w_{1} \bigoplus w_{2}) \bigoplus w_{3}~=~w_{1} \bigoplus (w_{2} \bigoplus w_{3}),~\forall w_{1}, w_{2}, w_{3}\in W$
    \item \emph{ Kommutatív:} $w_{1} \bigoplus w_{2}~=~w_{2} \bigoplus w_{1},~\forall w_{1}, w_{2}\in W$
    \end{itemize}
    \item \emph{ $\preceq$ teljes rendezés $W$-n:}
    \begin{itemize}
    \item \emph{ Reflexív:} $w \preceq w,~\forall w \in W$
    \item \emph{ Anti-szimmetrikus:} Ha $w_{1} \preceq w_{2}$ és $w_{2} \preceq w_{1}$, akkor $w_{1} ~=~ w_{2},~\forall w_{1}, w_{2} \in W$
    \item \emph{ Tranzitív:} Ha $w_{1} \preceq w_{2}$ és $w_{2} \preceq w_{3}$, akkor $w_{1} \preceq w_{3}~\forall w_{1}, w_{2}, w_{3} \in W$
    \item \emph{ Teljes:} $\forall w_{1}, w_{2} \in W$: $w_{1} \preceq w_{2}$ vagy $w_{2} \preceq w_{1}$
    \end{itemize}
    \item \emph{ $\phi$ összeegyeztethető a ($W,~\bigoplus$) Abel-csoporttal $\preceq$ szerint:}
    \begin{itemize}
    \item \emph{ Elnyelés:} $w \bigoplus \phi ~=~ \phi, ~\forall w \in W$
    \item \emph{ Maximalitás:} $\phi \npreceq w, ~\forall w \in W$
    \end{itemize}
  \end{itemize}

  \begin{note}
    Kiemelném az összehasonlítás operátor ($\preceq$) \emph{teljes} rendezési tulajdonságát, mert az a tulajdonság teszi lineárissá\footnote{Az első három tulajdonság miatt csak részbenrendezett halmazról beszélhetünk, ha azonban minden elem összehasonlítható, akkor válik teljes, vagy lineáris rendezéssé $\preceq$.} a rendezést.
  \end{note}

  \begin{note}
    A $\phi$ végtelen elem puszta létét, ill. összeegyeztethetőségét az Abel-csoporttal szintén fontos kiemelni, hiszen ezáltal szinten bármilyen alaphalmazt megadhatunk az élek bármelyik, tulajdonság-vektorbeli dimenziójának. Emellett érdemes megjegyezni, hogy azon (részben)rendezett halmazok, melyek bármely kételemű részhalmazának létezik infimuma és szuprémuma, hálóknak nevezzük. A routing algebrák esetében az természetesen csak akkor teljesül, ha alkalmasan választjuk meg a rendező ($\preceq$) operátort és az alaphalmazt: a valós számhalmazon, a ,,hagyományos'' $\leq$ rendezés esetén a routing algebrák hálók.
  \end{note}

  Most már meg tudjuk mondani egy egyszerű él súlyát. Az útvonalválasztás során azonban nem éleket akarunk összehasonlítani, hanem útvonal-alternatívákat:

  \begin{definition} [Egy út súlya]
    Egy $p=(v_{1}, v_{2}, ..., v_{k})$ út $w(p)$ hosszát az út éleinek súlyainak $\bigoplus$-szerinti összege adja: $$w(p)~=~\bigoplus_{i=1}^{k-1}w(v_{i}, v_{i+1}).$$
  \end{definition}

  Ezután azt mondjuk, hogy egy preferált (tetsző, legjobb, stb.) út az $\mathcal{A}$ algebrában $s$ és $t$ között a legkisebb súlyú $\preceq$ szerint: $$Pol(\mathcal{P}_{st})~=~p^{*}:~w(p^{*})~\preceq~w(p),~\forall p \in \mathcal{P}_{st}.$$

  Ezek után könnyen ellenőrizhető, hogy a legáltalánosabban használt routing policy, a shortest path routing (legrövidebb utak) algebrája a ($\mathbb{R}^{+},~\infty,~+,~\leq$), míg egy másik policy, a widest-path routing (legszélesebb utak) algebrája a ($\mathbb{R}^{+},~0,~min,~\geq$).

  %----------------------------------------------------------------------------
  \section{Routing algebrák tulajdonságai}\label{section_algebratulajdonsagok}
  %----------------------------------------------------------------------------

  A routing algebráknak, mint matematikai struktúrának számos érdekes tulajdonsága van. Ezek közül vannak olyanok, melyek alapvetően befolyásolják az algebra felhasználhatóságát, hiszen az algebra olyan minőségbeli tulajdonságát határozzák meg, mint például az útvonalválasztás algoritmikus lépésszáma. Vannak pusztán leíró jellegű tulajdonságok is, melyek segítenek az algebrák összehasonlításában, az egyes policy-primitívek kiválasztásában is.

    %----------------------------------------------------------------------------
    \subsection{Monotonitás, izotonitás és egyéb tulajdonságok}
    %----------------------------------------------------------------------------

    A legtöbb esetben két meghatározó tulajdonsággal kell rendelkeznie egy algebrának, hogy ,,jól viselkedőnek'' mondjuk. Az ilyen algebrákat reguláris algebrának nevezzük:

    \begin{definition} [Reguláris algebra]
      Az $\mathcal{A}$ routing algebrát regulárisnak nevezzük, ha
      \begin{itemize}
      \item \emph{ Monoton (M):} $w_{1} \preceq w_{1} \bigoplus w_{2}, ~\forall w_{1}, w_{2} \in W$
      \item \emph{ Izotón (I):} $w_{1} \preceq w_{2}~\Rightarrow~w_{3} \bigoplus w_{1} \preceq w_{3} \bigoplus w_{2},~\forall w_{1}, w_{2}, w_{3} \in W$
      \end{itemize}
    \end{definition}

    A monotonitás (M) azt jelenti, hogy egy $w_1$ súlyú élet, elé illesztve egy $w_2$ súlyú másik éllel csak kevésbé preferáltabbá teheti: $w_{1} \preceq w_{2} \bigoplus w_{1}$. A kommutativitás miatt igaz az él után való illesztésre is: $w_{1} \preceq w_{1} \bigoplus w_{2}$. A hagyományos értelemben vett hosszúságot általánosítja ez a tulajdonság, azaz a ,,a hosszabb út rosszabb''\footnote{Pontosabban a hosszabb út nem jobb, mint a rövidebb.}.

    Az izotonitás (I) azt jelenti, hogy adott $\preceq$ relációban álló éleket ugyanazzal az éllel elölről, vagy hátulról meghosszabbítunk, akkor a relációs viszony nem változik (Az itt leírtak igazak élek helyett utakra is).\\

    \Aref{routingmodell}. függelékben megtalálható a routing teljes modellje, melyet a routing algebrák tervezésekor használtak \cite{Sobrinho_Network_routing}. Ez a modell a lokális memóriaigényre koncentrál, és azt mondja, hogy egy $\mathcal{A}$ routing algebra tömöríthetetlen, ha az $\mathcal{M_{\mathcal{A}}}$ lokális memóriaigény $\Omega(n)$, különben $\mathcal{A}$ tömöríthető. Egy tömöríthetetlen routing algebra nyilván nem skálázódik jól, viszont a tömöríthető algebrák igen. A reguláris algebrák ,,jól viselkednek'', hisz a monotonitás és izotonitás garantálja, hogy a preferált út megkapható polinom időben az általánosított Dijkstra algoritmussal. Ez lehetővé teszi, hogy legfeljebb $\tilde{O}(n)$ bit információt tároljunk lokálisan, így nem csak az elméletben, hanem a valóságban is használható algebrákat kaphatunk \cite{Sobrinho_Network_routing, Sobrinho_Metarouting}.\\

    A teljesség igénye nélkül felsorolok még néhány tulajdonságot \cite{Lexicographic_products_in_metarouting}:
    \begin{itemize}
      \item \emph{ Delimitált (D):} $w_{1} \bigoplus w_{2} ~\neq~\phi, ~\forall w_{1}, w_{2} \in W$
      \item \emph{ Szigorúan monoton (SM):} $w_{1} \prec w_{1} \bigoplus w_{2}, ~\forall w_{1}, w_{2} \in W$
      \item \emph{ Kiválasztó (S):} $w_{1} \bigoplus w_{2} \in \{w_{1}, w_{2}\}, ~\forall w_{1}, w_{2} \in W$
      \item \emph{ Elnyelő (C):} $w_{1} \bigoplus w_{2} ~=~ w_{1} \bigoplus w_{3}, ~\forall w_{1}, w_{2}, w_{3} \in W$
    \end{itemize}

    A fentiek közül talán csak a delimitáltság (D) szorul magyarázatra. Ez a tulajdonság garantálja, hogy az éleket bármilyen önkényesen választott sorrendben is kombinálva járható utat kapunk. Általában az intra-domain\footnote{Intra-domain: AS-n belüli.} routing policy-k rendelkeznek ezzel a tulajdonsággal, de vegyük csak példának a BGP völgymentességét: ha egy csomag a hierarchiában fentebbi AS-től érkezik, akkor azt már csak lefelé mutató, vagy peer kapcsolaton keresztül lehet továbbítani.

  %----------------------------------------------------------------------------
  \section{Műveletek routing algebrák között}\label{section_algebramuveletek}
  %----------------------------------------------------------------------------

  A routing algebrák megadási módja igen változatos policy-k leírására ad lehetőséget, de - annak érdekében, hogy teljes rendezés lehessen $\preceq$ - mindig csak egy metrika szerint tudunk optimalizálni. Ha szeretnénk több szempontot is figyelembe venni, amire van is példa a routing policy-k között, akkor ezt az algebrák egymásutáni alkalmazásával tudjuk megtenni, így néhány egyszerű policy algebrájával meglepően bonyolult és a valóságot nagyon jól leíró algebrákat tudunk létrehozni \cite{Sobrinho_Metarouting}. Két ilyen műveletet igen fontos szerepet kap, az egyik a lexikografikus szorzat, ami az összeillesztő-, és a subalgebra, ami a szétválasztó operátor \cite{Lexicographic_products_in_metarouting}.

  \begin{definition} [$\mathcal{A}~\times~\mathcal{B}$]
    Legyenek adottak az $\mathcal{A}~=~(W_{\mathcal{A}},\phi_{\mathcal{A}},\bigoplus_{\mathcal{A}}, \preceq_{\mathcal{A}})$ és\\ $\mathcal{B}~=~(W_{\mathcal{B}},\phi_{\mathcal{B}},\bigoplus_{\mathcal{B}},\preceq_{\mathcal{B}})$ algebrák. Ekkor az $\mathcal{A}~\times~\mathcal{B}~=~(W,\phi,\bigoplus,\preceq)$ lexikografikus szorzatuk, ahol
    \begin{itemize}
    \item $W~=~W_{\mathcal{A}}~\times~W_{\mathcal{B}}$, ill. $\phi ~=~ (\phi_{\mathcal{A}}, \phi_{\mathcal{B}})$
    \item $(w_{1},v_{1}) \bigoplus (w_{2},v_{2})~=~ (w_{1} \bigoplus_{\mathcal{A}} w_{2},v_{1} \bigoplus_{\mathcal{B}} v_{2}),~\forall w_{1},w_{1} \in W_{\mathcal{A}}$ és $v_{1}, v_{2} \in W_{\mathcal{B}}$
    \item $(w_{1},v_{1}) \preceq (w_{2}, v_{2})~=~
    \begin{cases}
      v_{1} \preceq_{\mathcal{B}} v_{2} & \text{ha } w_{1}~=_{\mathcal{A}}~w_{2} \\
      w_{1} \preceq_{\mathcal{A}} w_{2} & \text{különben}
    \end{cases}$
    \end{itemize}
  \end{definition}

  \begin{note}
    A $\phi$ jól definiált, ha $\mathcal{A}$ és $\mathcal{B}$ delimitáltak. Egyéb esetben $\phi$ megadása nem magától értetődő.
  \end{note}
  \begin{note}
    A lexikografikus szorzat nem kommutatív: $\mathcal{A}~\times~\mathcal{B}~\neq~\mathcal{B}~\times~\mathcal{A}$.
  \end{note}

  A korábban már említett shortest-path és a widest-path policy-k algebráinak lexikografikus szorzata az $\mathcal{SW}$ (shortest-widest - legrövidebb-legszélesebb routing: a legszélesebb útra irányít, ám ha több ilyen is van, akkor azok közül a legrövidebben) és a $\mathcal{WS}$ (widest-shortest - legszélesebb-legrövidebb routing: először a legrövidebb útra irányít, ha több ilyen is van, akkor azok közül a legszélesebben).\\

  A második definiált művelet a subalgebra:
  \begin{definition} [$\hat{\mathcal{A}}$]
    Adott az $\mathcal{A}~=~(W,\phi,\bigoplus,\preceq)$ algebra és a súlyok egy $\hat{W}~\subseteq~W$ részhalmaza. Az $\mathcal{A}$ algebra leszűkítése $\hat{W}$-re: $\hat{\mathcal{A}}~=~(\hat{W},\phi,\bigoplus,\preceq)$ akkor és csak akkor subalgebrája $\mathcal{A}$-nak, ha $\hat{W}$ zárt $\bigoplus$-ra.
  \end{definition}

    %----------------------------------------------------------------------------
    \subsection{A műveletek hatása a tulajdonságokra}\label{section_algebramuveletek_tulajdonsagok}
    %----------------------------------------------------------------------------

    A routing algebrák kompozíciójaként létrejött új algebrák tulajdonságai levezethetők az alkotó algebrákéból.

    A következő tétel mutatja a lexikografikus szorzat hatását a tulajdonságokra:
    \begin{theorem}[A lexikografikus szorzat hatása az algebrák tulajdonságaira.]\label{eq:lexi}
      .
      \begin{itemize}
      \item $M(\mathcal{A} \times \mathcal{B})~\Leftrightarrow~ SM(\mathcal{A}) \bigvee (M(\mathcal{A}) \bigwedge M(\mathcal{B}))$
      \item $I(\mathcal{A} \times \mathcal{B})~\Leftrightarrow~ I(\mathcal{A}) \bigwedge I(\mathcal{B}) \bigwedge (N(\mathcal{A}) \bigvee C(\mathcal{B}))$
      \item $SM(\mathcal{A} \times \mathcal{B})~\Leftrightarrow~ SM(\mathcal{A}) \bigvee (M(\mathcal{A}) \bigwedge SM(\mathcal{B}))$\\
      \end{itemize}
    \end{theorem}

    Az \eqref{lexi} tétel szerint annak, hogy reguláris algebrákat hozzunk létre, szükséges feltétele, hogy csakis izonton algebrákat használjunk fel, sőt a monotonitás is megkövetelt mindkét tagra vagy az első tag szigorú monotonitása.

  %----------------------------------------------------------------------------
  \section{Példák algebrákra és kombinálásukra}\label{section_algebrapeldak}
  %----------------------------------------------------------------------------

  \Aref{tab:table_algebrapeldak} táblázatban néhány példát látunk a leginkább kutatott intra-domain routing policy-kra az algebráikkal és a legfontosabb tulajdonságaikkal. Az utolsót leszámítva az összes felsorolt algebra delimitált és reguláris (D, M, I).

  \begin{table}[ht]
    \footnotesize
    \centering
    \caption{Routing policy-k, algebráik és fontosabb tulajdonságaik.}
    \begin{tabular}{ | l | c | c |}
    \hline
    Policy & Algebra & Tulajdonság\\
    \hline
    Legrövidebb út & $\mathcal{S}~=~(\mathbb{R}^{+},~\infty,~+,~\leq)$ & SM, I\\
    Legszélesebb út & $\mathcal{W}~=~(\mathbb{R}^{+},~\infty,~min,~\geq)$ & S, I, M\\
    Legmegbízhatóbb út &  $\mathcal{R}~=~((0,1],~0,~*,~\geq)$ & SM, I\\
    Legszélesebb-legrövidebb út & $\mathcal{WS}~=~\mathcal{S}~\times~\mathcal{W}$ & SM, I\\
    Legrövidebb-legszélesebb út & $\mathcal{SW}~=~\mathcal{W}~\times~\mathcal{S}$ & SM, $\neg$I\\
    \hline
    \end{tabular}\label{tab:table_algebrapeldak}
  \end{table}

  A $\mathcal{W}$ a widest-path routing algebrát jelenti \cite{Quality_of_service_routing_for_supporting_multimedia_applications}. Ennél az algebránál egy él súlya a kapacitása adja, és nyilván egy több élből álló úton az út kapacitása megegyezik az útmenti legkisebb kapacitással. Emellett nyilván minél nagyobb egy út globális kapacitása, annál inkább preferált. A $\mathcal{W}$ algebra kiválasztó is és megadható a $(\mathbb{R}^{+},~\infty,~min,~\geq)$ négyessel (az \eqref{algebratetel1} tétel értelmében pedig tömöríthető is).

  A legmegbízhatóbb út algebra ($\mathcal{R}$) úgy értelmezhetjük, hogy az élek súlyát az a valószínűség adja, hogy az adott élen a csomag hibátlanul át tud menni, így nagyobb érték a jobb. Nyilvánvaló, hogy $\mathcal{R}$ tartalmaz egy szigorúan monoton (SM) subalgebrát: ha nem engedjük meg a biztos valószínűséget, akkor minden út biztosan romlik, ha hozzáveszünk egy újabb élet, azaz az értéke csökken, mert egy 1-nél kisebb számmal szorozzuk az eddig súlyt: $\hat{\mathcal{R}}~=~((0,1),~0,~*,~\geq)$.

  A legszélesebb-legrövidebb út (widest-shortest path) $\mathcal{WS}$ routing esetén a legrövidebb utak közül a legnagyobb kapacitásút választjuk \cite{Quality_of_service_based_routing_A_performance_perspective}, míg a legrövidebb-legszélesebb út (shortest-widest path) $\mathcal{SW}$ (\cite{Quality_of_service_routing_for_supporting_multimedia_applications, On_path_selection_for_traffic_with_bandwidth_guarantees}) routing esetén, éppen fordítva, a legnagyobb kapacitású utak közül a legrövidebbet választjuk. Ezeket az algebrákat megkaphatjuk a $\mathcal{S}$ és a $\mathcal{W}$ algebrák lexikografikus szorzataként \cite{Lexicographic_products_in_metarouting}.
  \begin{note}
    $\mathcal{SW}$ nem izotón. Az \eqref{algebratetel2} tétel áll a nem izotón algebrákra is, így $\Omega(n)$ bit lokális memóriát igényel a $\mathcal{SW}$ algebra is. Jelenleg még nyitott kérdés, hogy ez a határ szoros-e. Nem tudjuk, hogy van-e jobb megoldás, mint a triviális, azaz hogy minden router tárol egy-egy routing tábla bejegyzést minden forrás-cél párra, ami $O(n^2 log d)$ ($d$ a maximális fokszám) bit per router memóriaigényű.
  \end{note}

  %----------------------------------------------------------------------------
  \section{Összefoglaló}\label{section_osszefoglalo1}
  %----------------------------------------------------------------------------

  Ebben a fejezetben áttekintettem a szakirodalmat, összeszedve a legfontosabb állomásokat. Az 1990-es évek végétől egyre többet foglalkoztak a kutatók a BGP hálózatának feltárásával, amivel el is jutottak addig, hogy van egy viszonylag pontos kép a hálózatról, de arra, hogy miért egy adott policy-t használnak az AS-ek, nem tudnak válaszolni. Rámutattam, hogy az Internet AS-szintű topológiáján kívül, a más tudományterületekről származó problémák útvonalválasztásáról sem tudunk sok mindent és ezért van szükség egy olyan eszközre, ami a policy-feltárás feladatát - általános esetben is - hatékonyan el tudja látni. Ehhez definiáltam a routing algebrát, bemutattam a legfontosabb tulajdonságait és műveleteit, emellett a legszélesebb körben használt policy-k algebráit is ismertettem.
\setcounter{note}{0}

%----------------------------------------------------------------------------
\chapter{A hálózatkutatás legfontosabb modelljei}\label{examples}
%----------------------------------------------------------------------------
Számos kutatás foglalkozik különböző hálózatok útvonalválasztási problémáival. Természetesen minden ilyen modellt le lehet írni a gráfelmélet eszközrendszerével, így a meghatározó policy-k azonosítása után vizsgálható \aref{modell}. fejezetben bemutatott algebrákkal. Ebben a fejezetben bemutatom a hálózatkutatás leggyakrabban vizsgált modelljeit, és leírom a modellekhez tartozó meghatározó policy-ket is.\\

Az útvonalválasztás szempontjából alapvető különbség van a lokális- és a globális optimalizálás között. Míg globális optimalizálás esetén az egész hálózatot figyelembe véve választjuk meg a legjobb útvonalat, addig lokális optimalizálásnál csak a helyi viszonyok számítanak. Jelenleg a számítógépes hálózatokban a két legelterjedtebben használt policy kategória a ,,distance-vector''\footnote{Távolság-vektor módszerek, a Bellman-Ford algoritmusra épül.} és a ,,link-state''\footnote{Link-állapot módszerek, a Dijkstra algoritmusra épül.}. A módszerek abban különböznek, hogy a hálózatról milyen információkat gyűjtenek és hogyan. Az összes csomópontról információt gyűjtő link-state típusú útvonalválasztáskor a hálózat megváltozásáról minden csomópontot értesíteni kell, hogy az új kapcsolatok alapján határozzák meg újra az összeköttetéseket. A distance-vector típusú útvonalválasztási szabályrendszerek esetében a hálózat megváltozásakor a változást érzékelő csomópont mindig csak a szomszédait értesíti. Világos, hogy a distance-vector algoritmusok kevesebb üzenetküldés mellett alacsonyabb számítási komplexitással megoldhatók, ám abban megegyezik a két különböző típus, hogy globális optimumra törekszenek, tehát pl. a legrövidebb utat keresik meg.\\
Ezzel szemben pl. a greedy routing\footnote{Mohó útvonalválasztás: A lokális helyzet optimumát választja.} csupán \textit{törekszik} elérni globális optimumot, ám könnyen elakadhat, ha nem javítunk az alapalgoritmuson\cite{Routing_with_Guaranteed_Delivery_in_Adhoc_Wireless_Networks}. Könnyen belátható, hogy a vírusterjedés-jellegű feladatok lokális optimumra törekszenek -- hiszen nem ismerjük az egész hálózatot, míg az Internetes útvonalválasztás globálisan optimalizál.\\

A másik szempont, ami meghatározó, hogy a csomópontok felül tudják-e bírálni a csomagok terjedését abban az értelemben, hogy a küldő eredeti célját meg tudja-e valósítani akkor is, ha az útvonal mentén valamelyik csomópont ezt nem akarja. Feltehetjük-e, hogy a hibamentes csomagtovábbítás elsődleges prioritás minden hálózati csomópontnak (pl. kölcsönös bizalom megteremtése érdekében), vagy az egyes csomópontok viselkedhetnek önkényesen?\\

Mivel ezen két tulajdonság nem zárja ki egymást, a legtöbbször azzal az esettel állunk szemben, hogy nem tudunk egy problémát egyértelműen karakterizálni.

  %----------------------------------------------------------------------------
  \section{Vírusterjedés komplex hálózatokban}
  %----------------------------------------------------------------------------
  A vírusok terjedését és járvánnyá fejlődését sok évtizede vizsgálják. A járványok előrejelzése lehetőséget ad a tudósoknak, hogy megtervezzék a védőoltások ütemezését és az esetleges karanténok felállítását, ami jelentős hatással lehet egy adott betegség halálozási rátájára. A fertőző betegségek modellezése egy olyan eszköz, ami segítséget ad a fertőzés terjedésének vizsgálatához, emellett a modellezés segítségével stratégiákat lehet kidolgozni a jövőbeli járványok elkerülése érdekében\cite{Epidemic_Modelling_An_Introduction}.

    %----------------------------------------------------------------------------
    \subsection{A vírusterjedés matematikai modellje}
    %----------------------------------------------------------------------------
    A legkorábbi ismert matematikai modell a fertőző betegségek terjedéséről a híres holland tudóstól, Daniel Bernoullitól származik. Bernoulli a fekete himlőt vizsgálva megállapította, hogy ha mindenkit beoltanának a betegség ellen, akkor az átlagos életkor 26 év 7 hónapról 29 év 9 hónapra nőne\cite{Bernoulli_Blower}.

    Bernoulli, munkája során természetesen még nem értette annyira a baktériumok és vírusok biológiáját, mint mi. A XX. század első felében Ronald Ross malária kutatásával kezdetét vette a modern elméleti fertőzés-kutatás. Ezután nem sokkal, 1927-ben A. G. McKendrick and W. O. Kermack elismert munkája is megjelent, az ,,A Contribution to the Mathematical Theory of Epidemics''. Ez a determinisztikus modell sikeresen jelezte előre több fertőző betegség viselkedését.\\

    Egy determinisztikus modellben a populáció egyedeit besorolhatjuk alcsoportokba, ami a fertőzés egy specifikus stádiumát jelenti. Amikor nagy populációkat vizsgálunk, a használjuk ezt a modellt, hiszen világos, hogy pl. a populáció egyedei nem egyszerre lesznek fertőzöttek. Ha a fertőzés átterjedése az egyik csoportról a másikra időben folytonos, a csoportok aktuális mérete matematikailag a deriválással fejezhető ki, így a modell leírható differenciál egyenletekkel. A determinisztikusság azért játszik fontos szerepet, mert egy ilyen modellben feltesszük, hogy egy csoport egyedszáma deriválható az idő szerint, ehhez viszont az kell, hogy a fertőződés kiszámíthatóan (nem véletlenszerűen) terjedjen tovább, más szóval a csoportok egyedszámának változása egyértelműen meghatározható az addigi múltból. Ezen modellek szakirodalmát és terminológiáját áttanulmányozva számos különböző mérőszámot és tulajdonságot találhatunk. Az alapmodell a következő három csoportot jelöli ki:
    \begin{itemize}
      \item \emph{S(t)}: A $t$ időpontig meg nem fertőzött egyedek száma, azaz akik még megfertőződhetnek.
      \item \emph{I(t)}: A $t$ időpontig megfertőzött egyedek száma, azaz akik tovább tudnak fertőzni.
      \item \emph{R(t)}: A $t$ időpontig megfertőzött, de mégsem I(t)-beli egyedek száma, azaz vagy meggyógyultak, vagy meghaltak. Ezen csoport egyedei már nem tudnak újra megbetegedni, sőt fertőzni sem fertőznek.
    \end{itemize}

    Egy egyed az $S(t) \rightarrow I(t) \rightarrow R(t)$ sorrendben halad a csoportok között. Egy fix méretű populációt feltételezve, $N~=~S+I+R$, a következő egyenleteket vezethetjük be\cite{Contributions_to_the_mathematical_theory_of_epidemics}:
    \begin{align}
      \frac{\partial S}{\partial t} &= - \beta S I\\
      \frac{\partial I}{\partial t} &= \beta S I - \gamma I\\
      \frac{\partial R}{\partial t} &= \gamma I
    \end{align}

    Ahol $\beta$ a kapcsolati ráta és $1/\gamma$ az átlagos fertőző periódus.\\

    Néhány feltételezést tettünk az egyenletek felírásához: (1) minden egyedet ugyanakkora valószínűséggel fertőz meg egy $\beta$ kapcsolati rátájú betegség, így minden fertőzött egyed $\beta N$ egészséges egyedet fertőz meg egy időegység alatt, illetve az egyedeknek átlagosan $S/N$ megfertőzhető kapcsolata van. Tehát a fertőzési ráta: az új betegek száma $\beta N (S/N)$, azaz amennyivel változik (nyilván csökken) $S$: $\beta N (S/N)I~=~\beta SI$.\\
    Ezután érthető a második és harmadik egyenlet: a fertőzöttek száma annyival nő, ahánnyal kevesebb egészséges lesz, illetve annyival fogy, ahány meghal vagy meggyógyul. A $\gamma$ jelöli az átlagos halálozási / felépülési rátát.\\
    Végül feltesszük, hogy a megfertőződés és a felépülés időben jóval gyorsabb folyamat, mint a születés és halálozás, így ezeket a faktorokat kihagyhatjuk a modellből.

    \begin{note}
      Természetesen számos következő szempontot figyelembe lehet még venni a modell mélyítése érdekében, ilyen például az előbb említett születés és halálozás; az $R \rightarrow S$ átmenet, azaz a fertőzésből való meggyógyulás után újra meg lehet betegedni; figyelembe lehet venni az úgynevezett lappangási időszakot, bevezetve egy újabb csoportot ($S \rightarrow \epsilon \rightarrow I \rightarrow R$); vagy a veleszületett betegséget is, azaz a fertőzött anyától elkapott betegséget is.\newline
      Jelen dolgozat szempontjából az alap SIR modell megfelelő.
    \end{note}

    %----------------------------------------------------------------------------
    \subsection{Vírusterjedés, mint útvonalválasztási probléma}
    %----------------------------------------------------------------------------
    A fertőző betegségek terjedése nem látszik útvonalválasztási kérdésnek sőt, semmilyen policy-t nem veszünk észre első ránézésre. Ennek a dolgozatnak a célja rejtett szabályok felismerése előre definiált policy primitívek segítségével, ezért talán némi magyarázatot igényel a problémakör vizsgálata.\\

    A fejezet bevezetőjében említett két meghatározó tulajdonság közül az önérdek-követés az, ami itt megjelenik. A köznyelvben elterjedt szóhasználattal szemben nem \textit{egy} vírus emberek közötti terjedéséről van szó, hanem vírus egyedek egyéni útvonalválasztásáról. Világos, hogy ilyen értelemben lokális optimalizálásról beszélhetünk, hiszen minden egyes fertőző egyed szeretne minél tovább élni újabb és újabb gazdatestet találva magának. Lévén itt nem a tipikus -- több lépésen keresztüli, pontpárok közötti -- routing-ról van szó, globális optimalizálási feladat legfeljebb az lehet, hogy minél több csomópont legyen fertőzött\footnote{Ha a számítógépes vírusokra gondolunk, akkor a vírus egyedek egyéni döntései mellett természetesen vagy egy globális célt elérni próbáló személy is.}. Minden csomópontból arra fertőz tovább a vírus egyed, amerre a legtöbb eséllyel fog továbbélni -- ez nyilván függ az adott vírus egyedtől is, azaz például a leggyengébb immunrendszerű ember felé. Ez alapján két policy-t mutatok be:

    \begin{itemize}
      \item Fertőzési-határ: A vírus egyed legjobb továbbfertőzési policy-je.
      \item Unió-fedés: A globális fertőzés policy-je.
    \end{itemize}

      %----------------------------------------------------------------------------
      \subsubsection{Fertőzési-határ}
      %----------------------------------------------------------------------------
      A Fertőzési-határ policy alapú útvonalválasztás garantálja a legjobb túlélési esélyt egy vírus egyed számára. Ennek a policy-nek az algebrája az $\mathcal{F}$ ($(0,1],~0,~\max,~\geq$). Ennél az algebránál az élek súlya (egy (0,1] intervallumbeli valós szám) azt adja meg, hogy azt a csomópontot, ahová vezet, milyen valószínűséggel tudjuk megfertőzni, így nyilván a nagyobb jobb, a 0 pedig azt jelenti, hogy a csomópont $R(t)$-beli, azaz már biztosan nem lehet megfertőzni. Mivel az útvonalválasztás lokálisan optimalizál, ezért két alternatív élet pusztán a súlyuk alapján a $\geq$ operátorral tudunk összehasonlítani, illetve a két él összeillesztésére a két él súlyának maximumát vesszük.

      %----------------------------------------------------------------------------
      \subsubsection{Unió-fedés}
      %----------------------------------------------------------------------------
      Az Unió-fedés policy szerinti útvonalválasztás célja, hogy lehetőleg olyan élen fertőzzön tovább az egyed, amit még nem használtak. Ennek a policy-nek a számítógépes vírusok tárgyalásánál van nagyobb szerepe, semmint a biológiai fertőzéseknél, hiszen ez alapján -- akár már fertőzött gépet is -- minden olyan élet saját felügyelet alá vehetünk, amit még nem találtunk meg, így például ügyelhetünk arra, hogy semelyik fertőzött gép ne tudja frissíteni a vírusirtójának a vírusdefiníciós adatbázisát.

      Az Unió-fedés policy algebrája az $\mathcal{U}$ ($\mathbb{N},~\infty,~f,~\leq$). Itt az élsúly egy természetes szám lehet és azt fejezi ki, hogy hányszor használták már fel az adott élet egy útvonalválasztásban. Mivel a cél az, hogy minél ritkábban (optimálisan soha nem) használt utakat használjunk, a $\infty$ jelenti az átjárhatatlanságot és a $\leq$ az összehasonlító operátor. A több élből álló út súlyát az $f$ összegző függvény adja:
      $$f(e_1,e_2)~=~
      \begin{cases}
        e_1+e_2 & \text{ha } e_1 \neq 0 \text{ és } e_2 \neq 0\\
        0 & \text{különben}
      \end{cases}$$
      Ezzel garantáljuk, hogy olyan esetben, amikor nem számít -- nincs már fel nem használt él -- akkor az élek súlyának összegét vesszük, de amikor van olyan él, amit még nem használtunk, akkor azt biztosan fogjuk használni (hacsak nincs egy másik alternatív 0 súlyú út), hiszen a két élű út súlya 0 lesz.

  %----------------------------------------------------------------------------
  \section{Trendterjedés közösségi hálózatokban}
  %----------------------------------------------------------------------------
  A közösségi hálózatok az elmúlt 10 év egyik legmeghatározóbb jelensége, új korszakot nyitott az emberi kapcsolatokban. A kapcsolatháló elemzés a szociometriából indult, de annál bonyolultabb, időben is változó, nagyobb közösségek vizsgálatára alkalmas módszertan. A kapcsolatháló elemzési módszerek alkalmasak egy nagyobb csoport klikkjeinek és alcsoportjainak felrajzolására és megjelenítésére is. A kapcsolathálózati megközelítés kevésbé hangsúlyozza az egyéni cselekvés szerepét a struktúrák létrehozásában, nem az egyéni szándékból, hanem a struktúrák belső feszültségeiből jön létre a cselekvés mozgástere. A struktúra -- pl. egy elterjed trend -- e szemléletmód számára nem egy közvetlenül megmutatkozó adottság, hanem sokkal inkább a kapcsolatok hálójából bonyolultan kibontható szerveződés.

  A hálózati megközelítés előnye, hogy az adott hálózati struktúrákat, közösségeken belül kialakuló attribútumokat dinamikus módon, a változásokon keresztül is képes vizsgálni: a kapcsolatok és a környezet folytonosan változó mozgása mentén. A hálózat elemzéssel komplex módon tudjuk leírni vizsgált közösségi hálózatok működését\cite{Csaba_Pal}.\\

  A trend- vagy divatterjedés egy szociális hálóban lehetséges, ahol az egyik csomóponttal szimbolizált egyén felvesz egy szokást, elkezd hordani egy bizonyos ruhadarabot (vagy márkát), vagy bármi olyat tesz, amit ,,az átlag'' nem. Ha az ismerősei -- a kapcsolati hálóban szomszédos csomópontok -- észreveszik ezt az újdonságot és megtetszik nekik is, átveszik a forrástól. A problémakör alapvető kérdése, hogy mi határozza meg, hogy elterjed-e egy trend és divattá válik-e az egész hálózatban, vagy sem. Malcolm T. Gladwell brit-kanadai újságíró, író 2000-ben megjelent könyvében\cite{The_Tipping_Point} számos példát olvashatunk, hogy egyes szituációkban mikor érkezett el a fordulópont, mikor vált általános divattá valami. Ilyen példa a Hush Puppy cipőgyártó cég, mely egyik pillanatról a másikra a csőd széléről világszerte ismert márkává vált. Gladwell azt állítja, hogy a kulcs az úgynevezett \textit{Összekötőkön} múlik. Egy összekötő tipikusan hatalmas társasági kapcsolatrendszerrel rendelkezik, bár ez a kapcsolatok többnyire gyenge kötelékek csupán. Emellett egy összekötő tipikusan több mikrovilágba és szubkultúrába is bejáratos. Az ilyen összekötőket kell megnyernie egy reklámnak és ennek szinte automatikus következménye, hogy elterjed a termék és divattá válik. Duncan Watts ausztrál fizikus-matematikus-szociológus megkérdőjelezte Gladwell állítását, szerinte ugyanis nem az a meghatározó, hogy egy divat az összekötőktől indul-e, hanem az, hogy a közösség (akár az egész társadalom, akár csak egy szűkebb baráti kör) készen áll-e az adott divat elterjedésére. Tehát nem a forrás számít és annak befolyása, hanem az, hogy általánosságban a közösség egésze hogyan reagál termékre.

    %----------------------------------------------------------------------------
    \subsection{Trendterjedés, mint útvonalválasztási probléma}
    %----------------------------------------------------------------------------
    A trendterjedés nagyon hasonlít a vírusterjedésre, hiszen maga a két eredeti modell megfeleltethető egymásnak. Természetesen a vírusterjedés esetén említett kiegészítéseket nem lehet egy az egyben átültetni és a trendterjedés terminológiájával leírni, de az alapmodell három csoportját ($S$, $I$, $R$) itt is azonosíthatjuk. Világos, hogy az optimalizálás lokalizáltsága erre a modellre is jellemző, hiszen nincs olyan divat, ami egy bizonyos emberre akar csak hatni, sokkal inkább mindenkire. Így ennek a problémakörnek is jellemző policy-je lehet az \textit{Unió-fedés}.

    Útvonalválasztási szempontból a lényeges különbség, hogy a trendterjedésnél a csomópontoknak van döntő szerepe, nem az éleknek. Itt nincs közös érdek, ami egy divat elterjedését segítené (vagy akar gátolná). Bárki mondhatja azt, hogy neki kifejezetten nem tetszik az elterjedő divat, ezért lebeszéli az ismerőseit róla. Ennek enyhébb kifejezése már az is, ha nem veszi át a divatot, így a közvetlen közelében megállítja terjedést. Nyilvánvaló, hogy -- annak ellenére, hogy az elterjedés kulcsa a társadalmi elfogadás -- egy \textit{Összekötő} nagyban hozzájárulhat a sikeres terjesztéshez, így preferálhatjuk ezen központi egyéneket. Emellett az is szempont lehet, hogy nem a csomópontok befolyását vesszük alapul, hanem pusztán a mennyiségüket és egy adott útvonalválasztáskor minél hosszabb útra törekszünk. Egy másik alkalmazható trendelterjedési-heurisztika, hogy ha két csomópont közötti élen sok divat terjedt át, akkor ezt az élet érdemes felhasználni. Sőt megfordítva is alkalmazható, ha egy kapcsolat mentén eddig kevés divat terjedt át, akkor valószínűleg a két személy kapcsolata nem olyan szoros, amin a divatok átterjednek, így érdemes elkerülni az ilyen éleket. Ezek alapján a következő policy-ket definiálom:

    \begin{itemize}
      \item Összekötő-keresés: A nagy súlyú csomópontok felé irányító policy.
      \item Korai-elfogadó-keresés: A mindent átvevő csomópontok felé irányító policy.
    \end{itemize}

      %----------------------------------------------------------------------------
      \subsubsection{Összekötő-keresés}\label{osszekoto_kereses}
      %----------------------------------------------------------------------------
      Ez a policy azon csomópontokat részesíti előnyben, amelyeket a legbefolyásosabbnak ítélünk. A vírus- és trendterjedés hasonlóságából adódóan itt is lokális optimalizálásról beszélhetünk, így mindig azt a szomszédot fogjuk választani, aminek a legnagyobb a fokszáma. Az Összekötő-keresés algebrája a $\mathcal{O}$ ($(1,d),~0,~\max,~\geq$). Az élek súlya:
      $\forall u,v \in V:~w(u,v)~=~\max(\hbox{deg}(u),\hbox{deg}(v))$, tehát az összekötött csomópontok fokszámai közül a nagyobb. A rendező operátor nyilván a $\geq$, hiszen minél a több egy él $w$ értéke, annál jobb. Az átjárhatatlanságot a 0 súly jelzi, míg több él súlyát a maximumuk határozza meg.

      %----------------------------------------------------------------------------
      \subsubsection{Korai-elfogadó-keresés}\label{korai_elfogado_kereses}
      %----------------------------------------------------------------------------
      A Korai-elfogadó-keresés policy azon éleket részesíti előnyben, melyeken az eddigi tapasztalat alapján sok divat átterjedt már. Ennek az az alapja, hogy egy ilyen él olyan személyek közötti kapcsolatot jelöl, akik bíznak egymás értékítéletében, így átvesznek egymástól viselkedési formákat. A policy algebrája a $\mathcal{K}$ ($\mathbb{N},~-1,~+,~\geq$). Az élek súlya azt jelzi, hogy eddig mennyi divat terjedt át rajta, így a több a jobb, és nyilván a -1 súlyú élen nem tudunk átmenni. Több él összeillesztését az élsúlyok összegével súlyozunk.

  %----------------------------------------------------------------------------
  \section{Útvonalválasztás az Interneten}
  %----------------------------------------------------------------------------
  A útvonalválasztás kutatói rendszerint az informatika tudomány, vagy a számításelmélet szakemberei, ezért a kutatási célok legnagyobb része számítógépes hálózatok útvonalválasztásának vizsgálata, javítása. Ezen a területen általában az Internet gerinchálózatát szokták vizsgálni, hiszen ott jelentkeznek azok a skálázhatósági- és menedzsmentproblémák, amik az Internet 1970-es évekbeli nem elég körültekintő tervezéséből erednek. Ezen hibák miatt azt mondhatjuk, hogy az Internet hibáinak toldozása-foldozása már nem elég és alapvetően új megoldások szükségesek.\\

  Két alapvetően eltérő koncepció létezik a számítógépes hálózatok útvonalválasztásában. A tapasztalatok alapján előre becsült forgalmi viszonyoknak megfelelő centralizált kialakítás, illetve az aktuális forgalmi helyzet állandó figyelése alapján, az annak legjobban megfelelő irányítás. Legtöbbször az utóbbit valósítják meg, méghozzá elosztott változatban. Az Internet tervezése során több olyan szempontot is figyelembe vettek, melyeket az előző két alfejezetben nem tudtunk tárgyalni. Ilyen például a routerekben lévő útvonaltáblák minél kisebb mérete (kisebb tár, olcsóbb és gyorsabban működő csomópont, illetve kisebb routing forgalom), emellett fontos szempont a routing forgalom minimalizálása, a robusztusság (hibás tábla esélyének minimalizálása: ,,fekete lyuk'', hurok, oszcilláció elkerülése) és végül az optimális útvonalak kijelölése (természetesen az út optimalitása az igénytől függ). Az egyik legfontosabb igény az elosztott működés, a decentralizáltság volt azért, hogy az útvonalválasztást a csomópontok végezzék és ne egy központi -- nagy jelentőségű -- csomópont.\\

  Ahogy fejlődött a világ, újabb feladatok és problémák jelentek meg, amelyet a szakemberek a hálózat tervezésekor még nem láttak előre. Olyan új technológiák jelentek meg, mint a mobiltelefonok és a hordozható számítógépek és velük együtt az igény a mobil routing-ra, azaz arra, hogy ne csak a számítógép, de az internet hozzáférés is legyen hordozható. Problémát jelent a multicast routing is, amikor egy adott csomagot több címzetthez szeretnénk eljuttatni. Ezen problémák hatékony megoldása kritikus feladat, alapvetően új megoldásokat kívánnak.

  Emellett pusztán a hálózat mérete is feszegeti az Internet teljesítőképességének jelenlegi határait. Mára az Internet óriásira nőtt és a 1990-es évek vége óta a mindennapi élet meghatározó részévé vált. 2000 és 2009 között 394 millióról 1,84 milliárd nőtt az Internet kapcsolattal rendelkezők száma\footnote{Market Information and Statistics, International Telecommunications Union, \url{http://www.itu.int/ITU-D/ict/statistics/}}. Ma már több, mint 3,04 milliárd\footnote{3.035.749.340 -- 2014. június, \url{http://www.internetworldstats.com/stats.htm} (2014. 12. 09.)} felhasználója van az Internetnek, és 2010-ben 12,5 milliárd Internetre csatlakoztatott eszköz volt. Ez azt jelenti, hogy minden élő emberre jutott 1,84 készülék. 2020-ra az Internetre kapcsolt eszközök száma elérheti az 50 milliárdot\cite{The_Internet_of_Things}.

  Útvonalválasztási szempontból az Internet routing egy klasszikus feladata, amikor a legszélesebb-, legrövidebb- és legmegbízhatóbb utakat keressük. Ezen policy-k algebráit már bemutattam, megtalálhatók \aref{tab:table_algebrapeldak}. táblázatban.

  \begin{figure}[!ht]
    \centering
    \includegraphics[width=70mm,keepaspectratio=true]{./figures/BGP_iranyitatlan.png}\hspace{5mm}
    \includegraphics[width=70mm,keepaspectratio=true]{./figures/BGP_iranyitott_labeled.png}
    \caption{A BGP egyszerűsített képe és a völgymentesség szerinti irányítás.}
    \label{fig:figure_BGP}
  \end{figure}

  Érdemes megvizsgálni a BGP szabályrendszerét: a BGP routing policy-je többszintű. A legalsó szint, a legalapvetőbb policy a völgymentesség. Ez azt jelenti, hogy az útvonalválasztásnál elsődleges szempont, hogy semelyik AS-nek ne kelljen fizetni olyan forgalomért, ami csak áthalad rajta. Ha a hierarchiában lefelé mutató éleket $l$-lel, a felfelé mutató éleket $f$-fel, a peering kapcsolatokat pedig $p$-vel jelöljük, akkor minden routing során kijelölt útvonal csak a következő alakban írható le: néhány (akár nulla) $f$ él, aztán maximum egy $p$ él, utána pedig néhány (akár nulla) $l$ él.\\

  Emellett, ha felhasználjuk a gráfbeágyazás és kompakt routing eredményeit, akkor egy új távolságfüggvénnyel és a csomópontok koordinátázásával megvalósítható az Internet tartományszintű gráfjában is egy elakadásmentes mohó útvonalválasztás\cite{DobreiBScSzakdolgozat}.

  \begin{figure}[h]
    \centering
    \includegraphics[width=70mm]{./figures/3-reg_disk.png}\hspace{5mm}
    \includegraphics[width=70mm,keepaspectratio=true]{./figures/3-reg_half_plane.png}
    \caption{A hiperbolikus sík Poincaré-féle diszk modellje és a felső félsík modellje\cite{Klein07}.}
    \label{fig:figure_hiperbolikusabrak}
  \end{figure}

  A $\mathbb{H}$ hiperbolikus síknak többféle szemléltető modellje is van, ám a két legelterjedtebb a felső félsík modell és a Poincaré diszk modell, amiket \aref{fig:figure_hiperbolikusabrak}. ábrán láthatunk is. A Poincaré modellben $\mathbb{H}$-t egy egységkörrel reprezentálják: $x^2~+~y^2~<~1$, a következő metrikával:
  \begin{align}
    ds^2~=~\frac{4(dx^2~+~dy^2)}{(1~-~x^2~-~y^2)^2}.
  \end{align}

  A felső félsík modellben $\mathbb{H}$-t a $\{\langle x,y\rangle ~|~y~>~0\}$ ponthalmaz írja le, ahol a metrika:

  \begin{align}
    ds^2~=~\frac{dx^2~+~dy^2}{y^2}.
  \end{align}

  Mindkét esetben a $\mathbb{H}$ pontjait komplex számokként kezeljük: $(x,y)~\in~\mathbb{R}^2:~z~=~x~+~yi$.

  Ezek alapján két új policy-t mutatok be:

  \begin{itemize}
    \item Hiperbolikus-távolság: Az elakadásmentes mohó útvonalválasztás policy-je.
    \item Völgymentesség: A BGP alapvető, elsődleges policy-je.
  \end{itemize}

      %----------------------------------------------------------------------------
      \subsubsection{Hiperbolikus-távolság}
      %----------------------------------------------------------------------------
      A hiperbolikus síkra ágyazott Internet gráf minden pontja egy $(x,y)$ koordinátapárral írható le. A policy algebrája a $\mathcal{H}$ ($\mathbb{R}^{+},~\infty,~f_{\mathbb{H}},~\leq$), ahol $f_{\mathbb{H}}$ egy viszonylag bonyolult függvény, definiálásához szükséges lenne több, itt nem részletezett ismeret a hiperbolikus geometria témaköréből (bővebben ld.\cite{Thurston97} 2. fejezetét).

      %----------------------------------------------------------------------------
      \subsubsection{Völgymentesség}
      %----------------------------------------------------------------------------
      A völgymentesség policy-nek az algebrája a $\mathcal{V}$ ($\{f,~l,~p\},\phi,\bigoplus_{\mathcal{V}},\preceq$), ahol a $\bigoplus_{\mathcal{V}}$ \aref{tab:szumma_tab}. táblázat szerinti\footnote{ Egy adott súlyú (típusú) úthoz hozzá akarnánk venni egy élet, akkor az út milyen súlyúvá (típusúvá) válna.}. Az előbbi szabály másik megfogalmazásban azt jelenti, hogy sem $l$-t, sem $p$-t nem követhet sem $p$, sem $f$.

      \begin{table}[ht]
        \footnotesize
        \centering
        \caption{A $\bigoplus_{\mathcal{V}}$\cite{Compact_Policy_Routing}.}
        \begin{tabular}{ c | c c c }
          $\bigoplus$ & $f$ & $l$ & $p$\\
          \hline
          $f$ & $f$ & $l$ & $p$\\
          $l$ & $\phi$ & $l$ & $\phi$\\
          $p$ & $\phi$ & $p$ & $\phi$\\
        \end{tabular}\label{tab:szumma_tab}
      \end{table}

  %----------------------------------------------------------------------------
  \section{Egyéb algebrák}
  %----------------------------------------------------------------------------
  A minél részletesebb vizsgálat érdekében definiálok olyan policy-ket is, melyek vagy minden eddigi modellben használható lenne, vagy egyikben sem, így eddig nem volt lehetőségem bemutatni.

  A fejezet legelején felvázolt két dimenzió (optimalizálás, önérdekkövetés) mellett a harmadik karakterizálási lehetőség az időbeli lefolyás. Ha egy útvonalválasztási probléma tárgyalása során figyelembe vesszük az időt is, mint befolyásoló tényezőt, egy olyan új dimenziót ragadunk meg, mely minden modell routing-ját képes befolyásolni: ez az \textit{Időfüggés} policy. A vírus-terjedésnél elég csak arra gondolni, hogy télen könnyebben tud terjedni a fertőzés, és hasonlóan nyáron nagyobb valószínűséggel terjed egy fürdőruhadivat. Fontos azonban, hogy ez csak egy mellékes faktor, azaz egy alap policy mellett van értelme figyelembe venni az időpontot is. Az Internetes útvonalválasztás során időfüggést tapasztalhatunk, ha egy terheléselosztó rendszer egy adott kérést másodpercenként váltakozva egyszer kiszolgál, egyszer pedig újrapróbálkozásra szólít fel.\\

  Másik érdekes policy, a trendterjedésnél említett leghosszabb út policy. Ennek olyan esetben van értelme, amikor nem a célba érkezés a legfontosabb, hanem maga az út. A vírus- és trendterjedésnél ez azért volt lényeges, hogy minél több embert elérjünk, de van a természetből vett példa is: a hangyák is így építik ki a bolyban az utakat, hogy egy esetleges betolakodó minél nagyobb valószínűséggel eltévedjen.

      %----------------------------------------------------------------------------
      \subsubsection{Leghosszabb-út}
      %----------------------------------------------------------------------------
      A leghosszabb-út policy adja az elérhető leghosszabb utat. Ebben az esetben az $\mathcal{L}$ algebra az $(1, 0,~+,~\geq)$ négyes. Ebben a policy-ben minden él konstans 1 súlyú, egy út súlya éppen az élszáma.

      %----------------------------------------------------------------------------
      \subsubsection{Időfüggés}
      %----------------------------------------------------------------------------
      Az Időfüggés policy lényege, hogy az időpontot\footnote{Az idő-dimenzió léptéke problémafüggő: lehet naponkénti, de akár éves periódus is.} figyelembe véve néha átjárhatatlan egy-egy él. Ehhez szükséges egy alap policy is, amelyet ez ki tud egészíteni: $\mathcal{A}$. Jelölje $T_{e_i}$ az időpontok egy olyan halmazát, melyben $\mathcal{I}$ nem enged át forgalmat az $e_i$ élen. (A $T_{e_i}$ lehet akár egy $[t_0, \infty)$ intervallum is.) Így az $\mathcal{I}$ algebra: $(W_{\mathcal{A}},~\phi_{\mathcal{A}},~f,~\preceq_{\mathcal{A}})$, ahol
      $$f(e_1,e_2,t)~=~
      \begin{cases}
        \phi_{\mathcal{A}} & \text{ha } t \in T_{e_1} \cup T_{e_2}\\
        e_1 \bigoplus_{\mathcal{A}} e_2 & \text{különben}
      \end{cases}$$

      Azaz, alkalmas időpontban teljes egészében az $\mathcal{A}$ policy érvényesül, azonban minden élen bizonyos időpontokban nem lehet átmenni.\\

  %----------------------------------------------------------------------------
  \section{Összefoglaló}
  %----------------------------------------------------------------------------
  Ebben a fejezetben megvizsgáltam a hálózatkutatás szempontjából alapvető modelleket, melyeket a lokális- vagy globális optimalizálás és a közös- vagy egyéni érdekek követése tulajdonságok alapján karakterizáltam. Bemutattam a fertőző betegségek vizsgálatára használt matematikai modellt, megvizsgáltam a témakör útvonalválasztási kérdéseit és kijelöltem a két, a problémakört jól jellemző policy-t: a Fertőzési-határ és az Unió-fedés policy-ket, illetve ezek algebráit: $\mathcal{F}$ = ($(0,1],~0,~\max,~\geq$) és $\mathcal{U}$ = ($\mathbb{N},~\infty,~f,~\leq$).

  Rávilágítottam a trend- és a vírusterjedés hasonlóságaira és különbségeire útvonalválasztási szempontból és definiáltam két policy-t, a Összekötő-keresés-t és a Korai-elfogadó-keresés-t. Ezen policy-k algebrái: $\mathcal{O}$ = ($(1,d),~0,~\max,~\geq$) és $\mathcal{K}$ = ($\mathbb{N},~-1,~+,~\geq$).

  Végül megvizsgáltam a már ismertetett policy-ken (ld. \aref{section_algebrapeldak}. alfejezetet) kívül az Internet tartományszintű gráfjának az alapszabályát, a Völgymentességet, illetve felvázoltam a hiperbolikus térbe ágyazás -- általa pedig az elakadásmentes mohó útvonalválasztás -- lehetőségét: $\mathcal{V}$ = ($\{f,~l,~p\},\phi,\bigoplus_{\mathcal{V}},\preceq$) és $\mathcal{H}$ = ($\mathbb{R}^{+},~\infty,~f_{\mathbb{H}},~\leq$).

%----------------------------------------------------------------------------
\chapter{Val�s h�l�zatok vizsg�lata}\label{sect:chapter_test}
%----------------------------------------------------------------------------

 % \begin{figure}[!ht]
 % \centering
 % \includegraphics[width=67mm, keepaspectratio]{figures/TeXnicCenter.png}\hspace{1cm}
 % \includegraphics[width=67mm, keepaspectratio]{figures/TeXnicCenter.png}\\\vspace{5mm}
 % \includegraphics[width=67mm, keepaspectratio]{figures/TeXnicCenter.png}\hspace{1cm}
 % \includegraphics[width=67mm, keepaspectratio]{figures/TeXnicCenter.png}
 % \caption{T�bb k�pf�jl beilleszt�se eset�n t�rk�z�ket is �rdemes haszn�lni.} 
 % \label{fig:HVSpaces}
 % \end{figure}

 % \tabref{TabularExample}~t�bl�zat
 % \begin{table}[ht]
	 % \footnotesize
	 % \centering
	 % \caption{Az �rajel-gener�tor chip �rajel-kimenetei.} \label{tab:SysClocks}
	 % \begin{tabular}{ | l | c | c |}
	 % \hline
	 % �rajel & Frekvencia & C�l pin \\ \hline
	 % CLKA & 100 MHz & FPGA CLK0\\
	 % CLKB & 48 MHz  & FPGA CLK1\\
	 % CLKC & 20 MHz  & Processzor\\
	 % CLKD & 25 MHz  & Ethernet chip \\
	 % CLKE & 72 MHz  & FPGA CLK2\\
	 % XBUF & 20 MHz  & FPGA CLK3\\
	 % \hline
	 % \end{tabular}
	 % \label{tab:TabularExample}
% \end{table}

 % \begin{align}
 % \dot{\mathbf{x}}&=\mathbf{A}\mathbf{x}+\mathbf{B}\mathbf{u},\nonumber\\
 % \mathbf{y}&=\mathbf{C}\mathbf{x}\nonumber.
 % \end{align}

 % \begin{align}
 % \begin{bmatrix}
 % a_{11} & a_{12} & \dots & a_{1n}\\
 % a_{21} & a_{22} & \dots & a_{2n}\\
 % \vdots & \vdots & \ddots & \vdots\\
 % a_{m1} & a_{m2} & \dots & a_{mn}
 % \end{bmatrix}
 % \begin{pmatrix}x_1\\x_2\\\vdots\\x_n\end{pmatrix}=
 % \begin{pmatrix}b_1\\b_2\\\vdots\\b_m\end{pmatrix}.
 % \end{align}

 % \begin{align}
 % W(s)=\frac{A}{1+2T\xi s+s^2T^2}.
 % \end{align}

% \begin{lstlisting}[frame=single,float=!ht,caption=P�lda sz�veges irodalomjegyz�k-adatb�zisra BiBTeX haszn�lata eset�n., label=listing:Bibtex]
% asdfasdf
% asdf
% asdf
% afsdf
% \end{lstlisting}

% A diplomatervsablon (a kari ir�nyelvek szerint) az al�bbi f� fejezetekb�l �ll:
 % \begin{enumerate}
	 % \item 1 oldalas \emph{t�j�koztat�} a szakdolgozat/diplomaterv szerkezet�r�l (\verb+guideline.tex+), ami a v�gs� dolgozatb�l t�rlend�,
 % \end{enumerate}
%----------------------------------------------------------------------------
\chapter{�sszefoglal�s}\label{sect:chapter_test}
%----------------------------------------------------------------------------

%----------------------------------------------------------------------------
\chapter{Összefoglalás}\label{summary}
%----------------------------------------------------------------------------

A munka első felét elvégeztem, áttekintettem a hálózatkutatás eredményeit, különös tekintettel az útvonalválasztás matematikai kérdésére.\\

\Aref{modell}. fejezetben áttekintettem a szakirodalmat, összeszedve a legfontosabb állomásokat. Rámutattam, hogy az Internet AS-szintű topológiáján kívül, a más tudományterületekről származó problémák útvonalválasztásáról sem tudunk sok mindent és ezért szükséges egy olyan eszköz, ami a policy-feltárás feladatát - általános esetben is - hatékonyan el tudja látni. Ehhez definiáltam a \emph{routing algebrát}, bemutattam a legfontosabb tulajdonságait és műveleteit, emellett a legszélesebb körben használt policy-k algebráit is ismertettem.\\

\Aref{examples}. fejezetben megvizsgáltam a hálózatkutatás szempontjából alapvető modelleket, melyeket a lokális- vagy globális optimalizálás és a közös- vagy egyéni érdekek követése tulajdonságok alapján karakterizáltam. Bemutattam a fertőző betegségek vizsgálatára használt matematikai modellt, megvizsgáltam a témakör útvonalválasztási kérdéseit és kijelöltem a két, a problémakört jól jellemző policy-t. Emellett rávilágított a trend- és a vírusterjedés útvonalválasztási szempontbeli hasonlóságaira és különbségeire és definiáltam két új trendterjedési policy-t.\newline
Megvizsgáltam a már ismertetett policy-kon (ld. \aref{section_algebrapeldak}. alfejezet) kívül az Internet tartományszintű gráfjának az alapszabályát, a völgymentességet, illetve felvázoltam a hiperbolikus térbe ágyazás - általa pedig az elakadásmentes mohó útvonalválasztás - lehetőségét.\\

\Aref{framework}. fejezetben specifikáltam és megterveztem egy szimulációs keretrendszert, mellyel valós hálózati problémákon lehet tesztelni különböző útvonalválasztási stratégiák alapján leírt algebrákat. Definiáltam a szimulációhoz szükséges előfeldolgozási, adattisztítási lépéseket és magát a szimulációs folyamatot. Emellett kidolgoztam egy pontrendszert, mellyel értékelni lehet a szimulációs eredményeket. Keresztvalidációs módszerrel úgy alakítottam ki ezt a pontrendszert, hogy ne csak egy relatív skálát kapjak, ahol csak a vizsgált algebrákat hasonlíthatom össze egymással, hanem egy abszolút mércét is jelentsen az eredmény pontértéke.

\Aref{test}. fejezetben bemutattam a vizsgálandó valós hálózatot. Az adatok feldolgozása után meghatároztam a vizsgálandó hálózatot. Definiáltam a vizsgálandó metrikákat, amik mentén össze tudom hasonlítani a valós- és a policy vezérelt útvonalválasztás által meghatározott útvonalakat és hálózatokat. Minden pontpárra megvizsgálom az eredeti és a szimulált út különbségeit (távolság, lépésszám). A valós és a szimulált hálózatok statisztikai összehasonlítását is elvégzem, melyben figyelmet fordítok a fokszámeloszlás, a hálózatok átmérőjének és a pont- ill. él-összefüggőségének összehasonlítására.\\

A saját magam írt szoftverrel futtatott szimulációk eredményeinek alapján kijelenthető, hogy a diplomamunkámban tárgyalt keretrendszer - megfelelő körülmények között - alkalmas valós hálózati útvonalválasztási problémák vizsgálatára, algebrák tesztelésére.

\listoffigures\addcontentsline{toc}{chapter}{Ábrák jegyzéke}
%\listoftables\addcontentsline{toc}{chapter}{Táblázatok jegyzéke}

\bibliography{content/mybib}{}
\addcontentsline{toc}{chapter}{Irodalomjegyzék}
\bibliographystyle{my_bibtex_style}
%%----------------------------------------------------------------------------
% A bibtex m�g nem megy, addig is, j� lesz ez...
%----------------------------------------------------------------------------

\begin{thebibliography}{9}\addcontentsline{toc}{chapter}{Irodalomjegyz�k}

\bibitem{Lamport94} Leslie Lamport , \emph {\LaTeX : A Document Preparation System}.
Addison Wesley, Massachusetts, 2nd Edition, 1994.

\bibitem{Amport94} Amport , \emph {\LaTeX : A Document Preparation System}.
Addison Wesley, Massachusetts, 2nd Edition, 1994.

\end{thebibliography}

%----------------------------------------------------------------------------
\appendix
%----------------------------------------------------------------------------
\chapter*{\texorpdfstring{\protect\hypertarget{appendix}{Függelék}}{}}\addcontentsline{toc}{chapter}{Függelék}
\setcounter{chapter}{6}  % a fofejezet-szamlalo az angol ABC 6. betuje (F) lesz
\setcounter{equation}{0} % a fofejezet-szamlalo az angol ABC 6. betuje (F) lesz

\numberwithin{equation}{section}
\numberwithin{definition}{section}
\numberwithin{conjecture}{section}
\numberwithin{lemma}{section}
\numberwithin{theorem}{section}
\numberwithin{note}{section}
\numberwithin{figure}{section}
\numberwithin{lstlisting}{section}
%\numberwithin{tabular}{section}

  %----------------------------------------------------------------------------
  \section{Számításelméleti alapok}
  %----------------------------------------------------------------------------
  \begin{definition} [$O,~\Omega,~\Theta$]
    $f,g: \mathbb{N} \rightarrow \mathbb{R}$
    \begin{itemize}
    \item \emph{ Ordó:} $f(n)=O(g(n)),~ha~ \exists c > 0,~n_{0} > 0:~|f(n)| \leq c|g(n)|,~\forall n>n_{0}.$
    \item \emph{ Omega:} $f(n)=\Omega(g(n)),~ha~ \exists c > 0,~n_{0} > 0:~|f(n)| \geq c|g(n)|,~\forall n>n_{0}$
    \item \emph{ Teta:} $f(n)=\Theta(g(n)),~ha~ f(n)=O(g(n)) \text{ és } f(n)=\Omega(g(n)),~azaz~ \exists c_{1},c_{2} > 0,~n_{0} > 0:~c_{1}|g(n)| \leq |f(n)| \leq c_{2}|g(n)|,~\forall n>n_{0}$
    \end{itemize}
  \end{definition}

  \begin{note}
    A nagy ordós jelölésből szokás kihagyni a $logn$-es szorzót, minthogy egy decimális számot mindenképpen át kell konvertálni $logn$ bitre, így egy algoritmus lépésszámánál vagy (lokális / globális) memóriaigénynél biztosan nem tudjuk megspórolni: $\tilde{O}(n)~=~O(nlogn)$.
  \end{note}

%  \begin{definition} [K-szorosan pont-összefüggő gráf]
%    Egy G gráf k-szorosan pont-összefüggő, ha tetszőleges $k$-nál kisebb elemszámú csúcshalmazát elhagyva összefüggő gráfot kapunk, és $|V(G)| \geq k$.
%  \end{definition}
%  \begin{definition} [K-szorosan él-összefüggő gráf]
%    Egy G gráf k-szorosan él-összefüggő, ha tetszőleges $k$-nál kisebb elemszámú élhalmazát elhagyva összefüggő gráfot kapunk.
%  \end{definition}

%  Két alternatív definíció:

  \begin{definition} [K-szorosan pont-összefüggő gráf]
    Egy G gráf k-szorosan pont-összefüggő, ha tetszőleges $k$-nál kisebb elemszámú csúcshalmazát elhagyva összefüggő gráfot kapunk, és $|V(G)| \geq k$.
  \end{definition}

  \begin{definition} [K-szorosan él-összefüggő gráf]
    Egy G gráf k-szorosan él-összefüggő, ha tetszőleges $k$-nál kisebb elemszámú élhalmazát elhagyva összefüggő gráfot kapunk.
  \end{definition}

  \begin{definition} [Összefüggő gráf átmérője]
    A $G$ összefüggő gráf átmérője a két legtávolabbi csúcsának távolsága, azaz az összes csúcspár közötti legrövidebb utak közül a leghosszabbnak a hossza. Jelölje $P(u, v)$ az $u$ és $v$ csúcsok közötti utak halmazát, és $l(p)$ a $p$ út hosszát, ekkor a gráf átmérője:\\
    $$D_G = \max_{u,v \in V} \min_{p \in P(u,v)} l(p)$$
  \end{definition}

  \begin{definition} [K-reguláris gráf]
    Egy irányítatlan $G~=~(V,E)$ gráfot k-regulárisnak nevezünk, ha minden csúcsa pontosan k-fokú:
    $d(v)~=~k,~\forall v~\in~V.$
  \end{definition}

%  \begin{definition} [Síkba rajzolható gráf]
%    Egy gráf síkba rajzolható, ha a pontjainak kölcsönösen egyértelműen megfeleltethetők síkbeli pontok, az éleknek pedig a megfelelő csúcsokat összekötő egyenes szakaszok úgy, hogy a különböző szakaszok legfeljebb csak a végpontjaikban találkozhatnak.
%  \end{definition}

  \begin{definition} [DAG]
    Egy irányított gráf akkor DAG (Directed Acyclic Graph), ha nem tartalmaz irányított kört.
  \end{definition}

%  \begin{definition} [Topologikus rendezés]
%    A $G=(V,E)$ egy irányított gráf. $G$ egy topologikus rendezése, a csúcsok egy olyan $v_{1},~v_{2},...,~v_{|V|}$ sorrendje, melyben $i~\rightarrow~j~\in~E$ esetén $v_i$ előbb van, mint $v_j$.
%  \end{definition}

%  \begin{theorem} [Topologikus rendezés létezése]
%    Egy irányított gráfnak akkor és csak akkor van topologikus rendezése, ha DAG.
%  \end{theorem}

  \begin{theorem} [Maximális út létezése]\label{max_route}
    Egy irányított, súlyozott gráfban akkor lehet maximális út $u$ és $v$ között, ha a gráf DAG.
  \end{theorem}

  \begin{definition} [Gráfok izomorfiája]\label{grafizo}
    A gráfizomorfizmusok gráfok közötti bijektív struktúratartó leképezések, értve ezalatt azt, hogy a függvény és az inverz függvény egyaránt szomszédos csúcsokat szomszédos csúcsokra képez le:
    Adott $G(V,E)$ és $G'(V',E')$ gráfok. Egy $f:V \rightarrow V'$ bijektív függvény gráfizomorfizmus, ha $\{u,v\}\in E \Leftrightarrow \{f(u),f(v)\}\in E'$. Ilyenkor azt mondjuk, hogy $G(V,E)$ és $G'(V',E')$ izomorf.
  \end{definition}

  \begin{definition} [Eldöntési probléma]\label{eldontesi_def}
    Egy eldöntési problémához tartozó $L$ nyelv azoknak a bemeneteknek a halmaza, amelyekre a válasz IGEN. A lehetséges bemeneteket (amik tehát vagy beletartoznak $L$-be vagy nem), szavaknak hívjuk. Egy $X$ eldöntési probléma és $x$ bemenet esetén $x~\in~X$ jelöli, hogy az $x$ bemenetre a válasz IGEN.
  \end{definition}

  \newpage

  %----------------------------------------------------------------------------
  \section{Az útvonalválasztás teljes modellje }\label{routingmodell}
  %----------------------------------------------------------------------------
  Ahhoz, hogy felépíthessük a routing teljes modelljét, a routing algebrákon (\aref{section_routingalgebrak}. rész) kívül egy routing függvényre van szükség. Ebben a modellben a csomagok (ahogyan a valóságban is) hasznos teherből (payload) és egy header-ből állnak. Ha adott az $\mathcal{A}$ routing algebra és a G gráf, akkor a policy routing függvény az $R:~ \mathbb{N} \times \mathbb{N}~\mapsto~\mathbb{N} \times \mathbb{N}$ leképezés, a csúcsok $L_{V}:~V~\mapsto~\mathbb{N}$ címkézésével és az élek $L_{E}:~E~\mapsto~\mathbb{N}$ címkézésével, a következőképpen: minden $s,~t$ pontpárra egymás után alkalmazva R-t:
  $$(h_{i+1},~l_{i+1})~=~R(v_{i},~h_{i}),~\forall i~=~1, ..., k-1$$
  megadja a preferált $p_{st}^{*}~=~(s=) v_{1}, v_{2}, ..., v_{k}(=t)$ utat $\mathcal{A}$ szerint a megfelelő $l_{i+1}~=~(v_{i},v_{i+1})$ él-címkékkel, ahol $h_{1}$ egy alkalmas kezdő header. Azt mondjuk, hogy $R$ megvalósítja az $\mathcal{A}$ policy-t $G$-n. Még néhány óvintézkedést meg kell tennünk, hogy a címkékkel nehogy több routing információt kelljen kódolni a szükségesnél, azaz $c log n$\footnote{ Ez a címek kódolásához szükséges információ.}, valamilyen alkalmas $c$ konstanssal.\\

  Ezek alapján tehát a routing a következőképpen történik: Ha egy $u$ csomópont kap egy üzenetet $h$ header-rel, akkor egyszerűen kiszámolja az $R_{u}(h)$ lokális routing függvény értékét: $R_{u}(h)~=~R(u,h)$, hogy megkapja az új header-t, $h'$-t és a kimenő portot, $l$-t. Ezután $u$ beállítja a csomag header-jének $h'$-t és továbbküldi $l$-en keresztül.
  A routing függvénnyel már könnyen megadható egy hálózat minden csomópontjának a lokális memóriaigénye ahhoz, hogy egy adott routing policy-t valósítson meg:
  \begin{definition} [Routing policy megvalósításához szükséges lokális memória]
    Az $\mathcal{A}$ routing policy megvalósításához szükséges $\mathcal{M_{\mathcal{A}}}$ lokális memória: $$\mathcal{M_{\mathcal{A}}}~=~\max_{G \in \mathcal{G}_{n}}{ \min_{R \in \mathcal{R}}{\max_{u \in V}{ \mathcal{M_{\mathcal{A}}}(R,u)}}},$$ ahol $\mathcal{M_{\mathcal{A}}}(R,u)$ az $R_{u}$ lokális routing függvény kódolásához szükséges bitek minimális száma, $\mathcal{R}$ azon policy routing függvények halmaza, melyek megvalósítják $\mathcal{A}$ policy-t valamely $G$ gráfon, és $\mathcal{G}_{n}$ az összes $n$ csúcsú gráfok halmaza.
  \end{definition}

  %----------------------------------------------------------------------------
  \section{Tételek a routing algebrák témakörében}\label{algebratetelek}
  %----------------------------------------------------------------------------
  \begin{theorem} [Tömöríthetőség 1]\label{eq:algebratetel1}
    Ha $\mathcal{A}$ algebra kiválasztó (S) és monoton (M), akkor tömöríthető.
  \end{theorem}

  \begin{theorem} [Tömöríthetőség 2]\label{eq:algebratetel2}
    Ha $\mathcal{A}$ algebra szigorúan monoton (SM), akkor nem tömöríthető.
  \end{theorem}

  Ennél egy általánosabb tétel, aminek következménye az előző tétel:
  \begin{theorem} [Tömöríthetőség 3]\label{eq:algebratetel3}
    Ha $\mathcal{A}$ algebra tartalmaz egy delimitált (D), szigorúan monoton (SM) subalgebrát, akkor $\mathcal{A}$ nem tömöríthető.
  \end{theorem}
  \newpage

  %----------------------------------------------------------------------------
  \section{A szimulátor technikai részletei}\label{simulator}
  %----------------------------------------------------------------------------
  A szimulátort Java nyelven fejlesztettem, mivel semmilyen megkötés nem volt a \aref{framework}. fejezet specifikációjában, ami alapján ne lett volna alkalmas ez a megoldás. Azért, hogy ne csak a szimulációs eredményeket tudjam felhasználni a diplomamunkában, illetve ne csak egy elméleti megoldás legyen a keretrendszerem, törekedtem a szép, tiszta, érthető kód írására. Így mellékletként ez is értékelhető, a további munka során felhasználható.\\

  A fejlesztés során kér külső függvénykönyvtárat használtam fel: a Google Guava\footnote{A Google Guava forrása: \href{https://github.com/google/guava}{https://github.com/google/guava}} 18-as verzióját az adatszerkezetek és bejárásukra, illetve a Google Gson\footnote{A Google Gson forrása: \href{https://code.google.com/p/google-gson/}{https://code.google.com/p/google-gson/}} 2.2.4-es verzióját az eredmények JSON formátumban való mentéséhez és visszaállításukhoz.

  A szimulátor alapvetően három részre tagolódik: a hálózatot leíró osztályok, az algebrákat leíró osztályok és a szimulációt futtató osztályok.\\

  A \texttt{hu/dobrei/diploma/network package}-ben vannak az adott feladathoz tartozó hálózatot leíró osztályok, jelen esetben az \texttt{Airline}, \texttt{Airport}, \texttt{Flight}, \texttt{Route} és az \texttt{OpenFlightsNetwork} osztályok.\\

  A \texttt{hu/dobrei/diploma/algebra package}-ben van az \texttt{AbstractAlgebra} absztrakt osztály és a belőle származó \texttt{BusyAirportFindingAlgebra}, \texttt{LeastHopAlgebra}, \texttt{EarlyAdopterFindingAlgebra} és a \texttt{ShortestAlgebra} osztályok.\\

  A harmadik rész a \texttt{hu/dobrei/diploma/routing package}-ben levő osztályok, melyek a szimulációért, a végeredmények elemzéséért és az eredmények fájlba mentéséért felelősek.\\

  A tervezéstől kezdve a fejlesztése során végig arra törekedtem, hogy moduláris, könnyen újrahasználható szoftvert fejlesszek, melyet később tovább lehet adni. Ennek érdekében a lehető legtisztább, legegyszerűbb megoldásokat alkalmaztam, illetve ahol csak tudtam, ragaszkodtam a clean code\footnote{Robert C. Martin: Clean Code -- A Handbook of Agile Software Craftsmanship} elveihez.\\

  Összefoglalva elmondható, hogy sikerült elérni a célt a saját fejlesztésű szimulátorral: nem kellett másik szoftver használatát megtanulnom, biztos lehetek benne, hogy az útvonalválasztási algoritmus optimálisan van megírva és az algebrák implementálása pontos és egyszerű. Emellett további speciális felhasználásra viszonylag egyszerűen lehet módosítani a meglévő kódbázist.


\label{page:last}
\end{document}
























































 