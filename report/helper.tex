\newcommand\TODO[1]{\textcolor{red}{ TODO #1}}
\newcommand\todo[0]{\textcolor{red}{ TODO \newline}}

\makeatletter
\def\BState{\State\hskip-\ALG@thistlm}
\makeatother

\makeatletter
\renewcommand*{\ALG@name}{Algoritmus}
\makeatother

\pgfplotsset{
  /pgfplots/colormap={coldredux}{
    [1cm]
    rgb255(0cm)=(255,255,255)
    rgb255(2cm)=(0,192,255)
    rgb255(4cm)=(0,0,255)
    rgb255(6cm)=(0,0,0)
  }
}

\newcommand{\Histo}[3]{
  \begin{figure}[tbh]
    \centering
      \begin{subfigure}[b]{0.49\textwidth}
      \centering
      \resizebox {\textwidth} {!} {
        \begin{tikzpicture}
          \begin{axis}[
              xlabel={A csomópont befoka},
              ylabel={A csomópont kifoka},
              enlarge x limits=0.02,
              enlarge y limits=0.02,
              colorbar,
              colorbar style={
                %ylabel=A csomópontok aránya,
                yticklabel style={
                  text width=2.5em,
                  align=right,
                  /pgf/number format/.cd,
                  fixed,
                  %fixed zerofill
                }
              }
            ]
            \addplot[scatter, scatter src=explicit, only marks, mark=square*]
            file{sim/#1.data};
          \end{axis}
        \end{tikzpicture}
      }
      \caption{10-es csoportosítás}
    \end{subfigure} \begin{subfigure}[b]{0.49\textwidth}
      \centering
      \resizebox {\textwidth} {!} {
        \begin{tikzpicture}
          \begin{axis}[
              xlabel={A csomópont befoka},
              ylabel={A csomópont kifoka},
              enlarge x limits=0.02,
              enlarge y limits=0.02,
              colorbar,
              colorbar style={
                %ylabel=A csomópontok aránya,
                yticklabel style={
                  text width=2.5em,
                  align=right,
                  /pgf/number format/.cd,
                  fixed,
                  fixed zerofill
                  %{0, 0, , 0.1, , 0.2},
                }
              }
            ]
            \addplot[scatter, scatter src=explicit, only marks, mark=square*]
            file{sim/#1-zoom.data};
          \end{axis}
        \end{tikzpicture}
      }
      \caption{A maximum 30 ki- és befokú csúcsok.}
    \end{subfigure}
    \caption{#2 \label{#3}}
  \end{figure}
}

\newcommand{\AvgPlot}[3]{
  \begin{figure}[tbh]
    \centering
      \begin{subfigure}[b]{0.49\textwidth}
      \centering
      \resizebox {\textwidth} {!} {
        \begin{tikzpicture}
          \begin{axis}[
            xlabel={A csomópont átlagfokszáma},
            ylabel={A csomópontok aránya},
            enlarge x limits=0.02,
            enlarge y limits=0.02,
            legend pos=north west
          ]
          \addplot[smooth,color=blue]
          file{sim/#1-avg.data};
          \end{axis}
        \end{tikzpicture}
      }
      \caption{Az átlagfokszám-eloszlás.}
    \end{subfigure} \begin{subfigure}[b]{0.49\textwidth}
      \centering
      \resizebox {\textwidth} {!} {
        \begin{tikzpicture}
          \begin{axis}[
            xlabel={A csomópont átlagfokszáma},
            ylabel={A csomópontok aránya},
            enlarge x limits=0.02,
            enlarge y limits=0.02,
            yticklabels={0, 0, , 0.1, , 0.2},
            legend pos=north west
          ]
          \addplot[smooth,color=blue]
          file{sim/#1-avg-zoom.data};
          \end{axis}
        \end{tikzpicture}
      }
      \caption{A maximum 50 átlagfokszámú csomópontok.}
    \end{subfigure}
    \caption{#2 \label{#3}}
  \end{figure}
}
