\documentclass
[
compress,
%mathserif,
%handout,
t
]
 {beamer}

\usepackage[T1]{fontenc}
\usepackage[utf8]{inputenc}
\usepackage[magyar]{babel}
\usepackage[noend]{algpseudocode}
\usepackage{amsmath%,verbatim
	,hyperref}
\usepackage{url}
\usepackage{algorithm}
\usepackage{tikz, tikz-3dplot, pgfplots}
\pgfplotsset{compat=newest}
\usepackage{subcaption}
\usepackage{multirow}
\usepackage{graphicx}
\usepackage{ragged2e}


\uselanguage{hungarian}
\languagepath{hungarian}
\deftranslation[to=hungarian]{Theorem}{Tétel}
\deftranslation[to=hungarian]{Definition}{Definíció}
\deftranslation[to=hungarian]{Corollary}{Megjegyzés}

\usetheme{Darmstadt}
% AnnArbor, Antibes, Bergen, Berkeley, Berlin, Boadilla, CambridgeUS,
% Copenhagen, Darmstadt, default, Dresden, Frankfurt, Goettingen,
% Hannover, Ilmenau, JuanLesPins, Luebeck, Madrid, Malmoe, Marburg,
% Montpellier, PaloAlto, Pittsburgh, Rochester, Singapore, Szeged,
% Warsaw

\usecolortheme{beaver}
% albatross, beaver, beetle, crane, default, dolphin, dove, fly, lily,
% orchid, rose, seagull, seahorse, whale, wolverine

\usefonttheme{professionalfonts}
% default, professionalfonts, serif, structurebold,
% structureitalicserif, structuresmallcapsserif
\makeatother
\setbeamertemplate{footline}
{
  \leavevmode%
  \hbox{%
  \begin{beamercolorbox}[wd=.2\paperwidth,ht=2.25ex,dp=1ex,center]{author in head/foot}%
    \usebeamerfont{author in head/foot}\insertshortauthor
  \end{beamercolorbox}%
  \begin{beamercolorbox}[wd=.7\paperwidth,ht=2.25ex,dp=1ex,center]{title in head/foot}%
    \usebeamerfont{title in head/foot}\insertshorttitle\hspace*{3em}
  \end{beamercolorbox}
    \begin{beamercolorbox}[wd=.1\paperwidth,ht=2.25ex,dp=1ex,center]{title in head/foot}%
    \insertframenumber{} / \inserttotalframenumber\hspace*{1ex}
  \end{beamercolorbox}}%
  \vskip0pt%
}
\makeatletter
\setbeamertemplate{navigation symbols}{}

\newcommand\TODO[1]{\textcolor{red}{ TODO #1}}
\newcommand\todo[0]{\textcolor{red}{ TODO \newline}}

\makeatletter
\def\BState{\State\hskip-\ALG@thistlm}
\makeatother

\makeatletter
\renewcommand*{\ALG@name}{Algoritmus}
\makeatother

\pgfplotsset{
  /pgfplots/colormap={coldredux}{
    [1cm]
    rgb255(0cm)=(255,255,255)
    rgb255(2cm)=(0,192,255)
    rgb255(4cm)=(0,0,255)
    rgb255(6cm)=(0,0,0)
  }
}

\titlegraphic{\includegraphics[width=50mm,keepaspectratio]{figures/BMElogo.png}}
\title{Szabály alapú útvonalépítési stratégiák nagy hálózatokban}
%\subtitle{Diploma védés}
\author[Döbrei Gábor]{\large Döbrei Gábor (F3JY1C)\\ \footnotesize konzulens: Dr. Heszberger Zalán}
\institute[Budapesti Műszaki és Gazdaságtudományi Egyetem]{Budapesti Műszaki és Gazdaságtudományi Egyetem\\Távközlési és Médiainformatikai Tanszék\\MSc, Hálózatok és szolgáltatások szakirány}
\date{2015. január 15.}

\begin{document}

\begin{frame}
  \maketitle
\end{frame}

%----------------------------------------------------------------------------
\section*{A feladat}
%----------------------------------------------------------------------------
\begin{frame}{A feladatkiírás}
  \begin{itemize}
    \justifying
    \item Elemezze azokat a módszereket, melyek a \textbf{hálózatbeli kommunikációs folyamatokat alapvetően meghatározzák}!
    \item Állítson össze egy olyan lehetőség szerint \textbf{minimális primitív szabályrendszer halmazt}, mely segítségével a működés jól modellezhető!
    \item Az eredmények alapján \textbf{specifikáljon} és dolgozzon ki egy olyan \textbf{modellezési keretrendszert}, melynek segítségével nagyméretű \textbf{valós hálózatokat} jellemző \textbf{útvonalkialakítási dinamikát képes strukturált módon jellemezni}!
  \end{itemize}
\end{frame}

%----------------------------------------------------------------------------
% ToC
%----------------------------------------------------------------------------
\begin{frame}{Tartalom}
  \tableofcontents[pausesections]
\end{frame}

%----------------------------------------------------------------------------
% Slides
%----------------------------------------------------------------------------
%----------------------------------------------------------------------------
\section[Hálózatok]{Hálózatok és útvonalválasztási szabályok}
%----------------------------------------------------------------------------
  %----------------------------------------------------------------------------
  \subsection{Policy alapú útvonalválasztás}
  %----------------------------------------------------------------------------
  \begin{frame}[<+->]{Szabály alapú útvonalválasztás}

%    \begin{itemize}
%      \item Internet = világméretű IP hálózat (technológiai szemlélet)
%      \item Csomagkapcsolás
%      \item Routerekben az intelligencia
%      \item Lényegében az egyetlen szolgáltatás, amit az Internet nyújt!
%    \end{itemize}

    \begin{definition} [Routing policy]
      \justifying
      A routing policy egy olyan $Pol: \mathcal{P}_{st} \rightarrow p_{st}$ függvény, aminek az értelmezési tartománya a lehetséges $s - t$ utak: $\mathcal{P}_{st}$ és az adott policy-nek megfelelő legkedvezőbb utat adja vissza: $p_{st}^{*}=Pol(\mathcal{P}_{st})$.
    \end{definition}
  \end{frame}

  %----------------------------------------------------------------------------
  \subsection{Az útvonalválasztás matematikai modellje}
  %----------------------------------------------------------------------------
  \begin{frame}[<+->]{Routing algebrák}
    \begin{definition} [Routing algebra]
      \justifying
      Az $\mathcal{A}$ routing algebra egy teljesen rendezett félcsoport egy ,,végtelen elemmel'': $\mathcal{A}~=~(W,~\phi,~\bigoplus,~\preceq)$, ahol $W$ az élek súlyainak halmaza, $\phi~(\phi \notin W)$ egy speciális végtelen súly, abban az értelemben, hogy egy ilyen súlyú élen nem lehet átmenni, a $\bigoplus$ a súlyok kétváltozós kompozíció operátora, a $\preceq$ pedig a súlyok összehasonlító operátora.
    \end{definition}

    \begin{exampleblock} {$\mathcal{S}$ algebra}
      Shortest-path routing algebrája a $\mathcal{S} = (\mathbb{R}^{+},~\infty,~+,~\leq)$.
    \end{exampleblock}
  \end{frame}

  %----------------------------------------------------------------------------
  \subsection{Hálózatkutatási modellek}
  %----------------------------------------------------------------------------
  \begin{frame}{Hálózatkutatási modellek}
    \begin{itemize}
      \item<1-> Vírusterjedés komplex hálózatokban
      \begin{itemize}
        \item<4-> Fertőzési-határ algebra: $\mathcal{F} = ((0,1],~0,~\max,~\geq)$
        \item<4-> Unió-fedés algebra: $\mathcal{U} = (\mathbb{N},~\infty,~f,~\leq)$
      \end{itemize}
      \item<2-> Trendterjedés közösségi hálózatokban
      \begin{itemize}
        \item<5-> Összekötő-keresés algebra: $\mathcal{O} = ((1,d),~0,~\max,~\geq)$
        \item<5-> Korai-elfogadó-keresés algebra: $\mathcal{K} = (\mathbb{N},~-1,~+,~\geq)$
      \end{itemize}
      \item<3-> Útvonalválasztás az Interneten
      \begin{itemize}
        \item<6-> Hiperbolikus-távolság algebra: $\mathcal{H} = (\mathbb{R}^{+},~\infty,~f_{\mathbb{H}},~\leq)$
        \item<6-> Völgymentesség algebra: $\mathcal{V} = (\{f,~l,~p\},\phi,\bigoplus_{\mathcal{V}},\preceq$), ahol a $\bigoplus_{\mathcal{V}}:$
        \begin{table}
          \footnotesize
          \begin{tabular}{ c | c c c }
            $\bigoplus$ & $f$ & $l$ & $p$\\
            \hline
            $f$ & $f$ & $l$ & $p$\\
            $l$ & $\phi$ & $l$ & $\phi$\\
            $p$ & $\phi$ & $p$ & $\phi$\\
          \end{tabular}
        \end{table}
      \end{itemize}
    \end{itemize}
  \end{frame}


%----------------------------------------------------------------------------
\section{A modellezési keretrendszer}
%----------------------------------------------------------------------------

\iffalse
  %----------------------------------------------------------------------------
  \subsection{Tervezési szempontok}
  %----------------------------------------------------------------------------
  \begin{frame}{Tervezési szempontok}
    \todo
  \end{frame}
\fi

  %----------------------------------------------------------------------------
  \subsection{Pontrendszer}
  %----------------------------------------------------------------------------
  \begin{frame}{Pontrendszer}
    \begin{itemize}
      \item<1-> Pontpárok
      %A pontpárok mutatói -- amennyiben a globális mutatók nem romlottak el -- azt jelzik, hogy tudunk-e javítani az aktuális útvonalválasztáson, még a globális mutatók arról árulkodnak, hogy a generált gráf jellegében mennyire hasonlít a megfigyelt hálózatra
      \begin{itemize}
        \item<3-> Az algebra szerinti távolság: $AL$
        \item<4-> A lépésszám szerinti távolság: $HC$
      \end{itemize}
      \item<2-> Globális hálózati mutatók
      \begin{itemize}
        \item<5-> A fokszámeloszlás: $$DD =
    \begin{cases}
      1 & \text{ha megegyezik a fokszámeloszlás}\\
      \infty & \text{különben}
    \end{cases}$$
        \item<6-> A hálózat-átmérő: $GD$
        \item<7-> Az élösszefüggőség: $C$
        \item<8-> Az élszám pontozása: $EC$
      \end{itemize}
    \end{itemize}
  \end{frame}

  \begin{frame}[<+->]{A mutatók közötti kapcsolatok}
    \begin{itemize}
      \item $\frac{1}{(HC+1)} \cdot AL$
      \item $0 > \frac{\partial EC}{\partial C}$
      \item $0 > \frac{\partial GV}{\partial GD}$
      \item $0 > \frac{\partial GV}{\partial C}$
      \item $0 < \frac{\partial GV}{\partial EC}$
    \end{itemize}

    \onslide<6->$$GV = \frac{DD \cdot GD \cdot C}{ln(EC)}$$
  \end{frame}

  %----------------------------------------------------------------------------
  \subsection{A szimulátor}
  %----------------------------------------------------------------------------
  \begin{frame}{A szimulátor}
    %A szimulációs algoritmus a fent leírt irányított, súlyozott gráfon, mint bemeneten fut. Először meghatározzuk, hogy a konkrét probléma vizsgálata mekkora csúcshalmazt jelent, amelyből előáll egy $G(V=n, E=n \cdot (n-1))$ méretű gráf\footnote{Minden csúcsot minden csúccsal összekötünk és mindkét irányban irányítunk.}. Ezután meghatározzuk azon éleket, melyet ezen a gráfon az útvonalválasztási szabályrendszer akármilyen két pont között legalább egyszer használt. Ezt természetesen meg tudjuk tenni, hiszen a kialakult utakat megfigyelni képesek vagyunk. Miután meghatároztuk a felhasznált élhalmazt, minden olyan élet, amelyet nem tartalmaz ez a halmaz kitörlünk a gráfból (Az élek törlésével nem eshet szét a gráf, hiszen a ponthalmazt megfigyeléssel kaptuk, így biztos, hogy összefüggő marad a gráf.). A megmaradó éleket súlyozzuk.

    %A szimuláció menete ezután a következő: Minden pontpárra és minden vizsgált algebrára meghatározzuk (a módosított Dijkstra algoritmussal), hogy milyen ut(ak)at jelölnek ki. Az így meghatározott ut(ak)at feljegyezzük. Ezzel előállítunk két gráfot: az első az eredetileg vizsgált gráf, a másik a feljegyzett utakból felépülő. Ezután ezen két gráfot hasonlítjuk össze.
    \begin{itemize}
      \item<1-> Irányított, súlyozott gráfon
      \item<2-> $G(V=n, E=n \cdot (n-1))$ -- minden csúcspár oda-vissza irányítva
    \end{itemize}
    \onslide<3->\begin{algorithm}[H]
      \caption{Optimalizáló függvény}\label{algo_optimalizalo}
      \begin{algorithmic}[1]
        \Procedure{getPreferredPath}{$Vertex~u,~Vertex~v,~Algebra~a$}
          \State $\textit{paths} \gets \Call{getAllPathsBetween}{u,~v}$
          \State $\textit{preferred} \gets [~]$

          \For{\textbf{each} \textit{path} in \textit{paths}}
            \If {$a_W(path_i)~a_\preceq~a_W(path)$}
              \State $\textit{preferred} \gets \textit{preferred}~+~path$
            \EndIf
          \EndFor
        \EndProcedure
      \end{algorithmic}
    \end{algorithm}
  \end{frame}


%----------------------------------------------------------------------------
\section[Szimuláció]{A repülési hálózat vizsgálata}
%----------------------------------------------------------------------------
  %----------------------------------------------------------------------------
  \subsection{Adatok feldolgozása}
  %----------------------------------------------------------------------------
  \begin{frame}{Adatok feldolgozása}
    \todo
  \end{frame}

  %----------------------------------------------------------------------------
  \subsection{Szimulációs eredmények}
  %----------------------------------------------------------------------------
    \begin{frame}{Szimulációs eredmények}
      \center
      \only<1>{\resizebox {!} { 0.8\textheight } {
        \begin{tikzpicture}
          \begin{axis}[
            xlabel={A csomópont átlagfokszáma},
            ylabel={A csomópontok aránya},
            enlarge x limits=0.02,
            enlarge y limits=0.02,
            yticklabels={0, 0, , 0.1, , 0.2},
            legend pos=north west
          ]
          \addplot[smooth,color=blue]
          file{sim/least-avg-zoom.data};
          \end{axis}
        \end{tikzpicture}
      }}
      \only<2>{\resizebox {!} { 0.8\textheight } {
        \begin{tikzpicture}
          \begin{axis}[
              xlabel={A csomópont befoka},
              ylabel={A csomópont kifoka},
              enlarge x limits=0.02,
              enlarge y limits=0.02,
              colorbar,
              colorbar style={
                %ylabel=A csomópontok aránya,
                yticklabel style={
                  text width=2.5em,
                  align=right,
                  /pgf/number format/.cd,
                  fixed,
                  fixed zerofill
                  %{0, 0, , 0.1, , 0.2},
                }
              }
            ]
            \addplot[scatter, scatter src=explicit, only marks, mark=square*]
            file{sim/least-zoom.data};
          \end{axis}
        \end{tikzpicture}
      }}
    \end{frame}


%----------------------------------------------------------------------------
\section[Továbbfejlesztés]{Továbbfejlesztési lehetőségek}
%----------------------------------------------------------------------------
  \begin{frame}{Továbbfejlesztési lehetőségek}
    \begin{itemize}[<+->]
      \item Gyakorlat -- szimulátor
      \begin{itemize}
        \item Algebra leíró nyelv (Algebra Description Language)
        \item Automatizált beolvasás
        \item Szimuláció további optimalizálása
      \end{itemize}
      \item Elmélet -- kutatás
      \begin{itemize}
        \item Algebrák súlyozott összege
        \item Genetikus algoritmusok
        \item Pontozás általánosítás
      \end{itemize}
    \end{itemize}
  \end{frame}


%----------------------------------------------------------------------------
\section{Összefoglalás}
%----------------------------------------------------------------------------
  \begin{frame}{Összefoglalás}
    \begin{itemize}[<+->]
      \justifying
      \item Az útvonalválasztási szabályrendszerek felderítése kezdetleges volt, és nem állt a hálózati kutatók rendelkezésére olyan eszköz, amely a policy alapú routing felderítést lehetővé tette volna. A diplomamunkámban \textbf{megoldást adtam} erre a problémára.
      \item \textbf{Megterveztem egy szimulációs keretrendszert}, mely eleget tesz a feladatkiírásban szereplő elvárásoknak. Ennek keretében definiáltam egy pontrendszert, amellyel sikeresen össze lehet hasonlítani valós (megfigyelt) és szimulált hálózatokat.
      \item \textbf{\emph{Implementáltam}} a szimulátort és egy valós hálózaton bemutattam a működését. \textbf{Beazonosítottam} az $\mathcal{OS}$ algebrát, mint a legvalószínűbb policy-t, amely alapján a repülőtársaságok kialakíthatják útvonalaikat. Emellett \textbf{javaslatot tettem}, hogy az $\mathcal{OS}$ algebra helyett célszerűbb lenne az egyszerűbb $\mathcal{S}$ algebrát használni.
    \end{itemize}
  \end{frame}


%----------------------------------------------------------------------------
% Referee's Report
%----------------------------------------------------------------------------
%----------------------------------------------------------------------------
\section*{Bírálói kérdések}
%----------------------------------------------------------------------------
\begin{frame}{Bírálói kérdések}
  \begin{enumerate}
    \justifying
    \item Milyen megfontolások alapján került kiválasztásra a repülési adathalmaz más, publikusan elérhető adathalmazok közül?
    \item A szerző említi (F.4), hogy a routing algebrák optimális módon kerültek implementálásra, mindeközben cél volt a nagyméretű hálózatokban (ld. cím) való alkalmazhatóság. Fejtse ki, hogy a keretrendszer milyen memória, illetve algoritmikus skálázhatóságra képes!
    \item A szerző a házatokon belül kialakult útvonalválasztási stratégiák pontozására bevezetett globális mutatókat. Ezek révén kaptunk egy eszközt, amely nem csak a pont-pont utak milyenségét pontozza, hanem a használt útvonalak által kifeszített gráf milyenségét is, pl. hogy mennyire hibatűrő. Milyen méréseket, általánosításokat végezne, hogy a pontozási rendszer többféle hálózat (pl. internet routing topológia, tömegközlekedési forgalomszervezés) jellemzésére is konfigurálható legyen?
  \end{enumerate}
\end{frame}

\begin{frame}{Miért a repülési adathalmaz?}
  %Milyen megfontolások alapján került kiválasztásra a repülési adathalmaz más, publikusan elérhető adathalmazok közül?
  \begin{itemize}
    \item Gulyás András :) \todo
    \item Ne a BGP legyen
    \item Elég nagy legyen
  \end{itemize}
\end{frame}

\begin{frame}[<+->]{Memória, illetve algoritmikus skálázhatóság}
  %Fejtse ki, hogy a keretrendszer milyen memória, illetve algoritmikus skálázhatóságra képes!
  \begin{itemize}
    \item \todo
  \end{itemize}
\end{frame}

\begin{frame}[<+->]{A pontozási rendszer általánosítása, egyéb mérések}
  %Milyen méréseket, általánosításokat végezne, hogy a pontozási rendszer többféle hálózat (pl. internet routing topológia, tömegközlekedési forgalomszervezés) jellemzésére is konfigurálható legyen?
  \begin{itemize}
    \item \todo
  \end{itemize}
\end{frame}


%----------------------------------------------------------------------------
% Thank You
%----------------------------------------------------------------------------
\begin{frame}[c]
  \center
  \only<1>{\Huge Kérdések?}
  \only<2>{\Huge Köszönöm a figyelmet!}
\end{frame}


\end{document}