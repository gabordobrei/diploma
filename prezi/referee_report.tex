%----------------------------------------------------------------------------
\section*{Bírálói kérdések}
%----------------------------------------------------------------------------
\begin{frame}{Bírálói kérdések}
  \begin{enumerate}
    \justifying
    \item Milyen megfontolások alapján került kiválasztásra a repülési adathalmaz más, publikusan elérhető adathalmazok közül?
    \item A szerző említi (F.4), hogy a routing algebrák optimális módon kerültek implementálásra, mindeközben cél volt a nagyméretű hálózatokban (ld. cím) való alkalmazhatóság. Fejtse ki, hogy a keretrendszer milyen memória, illetve algoritmikus skálázhatóságra képes!
    \item A szerző a házatokon belül kialakult útvonalválasztási stratégiák pontozására bevezetett globális mutatókat. Ezek révén kaptunk egy eszközt, amely nem csak a pont-pont utak milyenségét pontozza, hanem a használt útvonalak által kifeszített gráf milyenségét is, pl. hogy mennyire hibatűrő. Milyen méréseket, általánosításokat végezne, hogy a pontozási rendszer többféle hálózat (pl. internet routing topológia, tömegközlekedési forgalomszervezés) jellemzésére is konfigurálható legyen?
  \end{enumerate}
\end{frame}

\begin{frame}{Miért a repülési adathalmaz?}
  %Milyen megfontolások alapján került kiválasztásra a repülési adathalmaz más, publikusan elérhető adathalmazok közül?
  \begin{itemize}
    \item Gulyás András :) \todo
    \item Ne a BGP legyen
    \item Elég nagy legyen
  \end{itemize}
\end{frame}

\begin{frame}[<+->]{Memória, illetve algoritmikus skálázhatóság}
  %Fejtse ki, hogy a keretrendszer milyen memória, illetve algoritmikus skálázhatóságra képes!
  \begin{itemize}
    \item \todo
  \end{itemize}
\end{frame}

\begin{frame}[<+->]{A pontozási rendszer általánosítása, egyéb mérések}
  %Milyen méréseket, általánosításokat végezne, hogy a pontozási rendszer többféle hálózat (pl. internet routing topológia, tömegközlekedési forgalomszervezés) jellemzésére is konfigurálható legyen?
  \begin{itemize}
    \item \todo
  \end{itemize}
\end{frame}
